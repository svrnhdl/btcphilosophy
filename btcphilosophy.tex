\hypertarget{dedication}{%
\section{Dedication}\label{dedication}}

To the relentless innovators and pioneers of Bitcoin development, whose
passion and commitment to building decentralized and trustless systems
have paved the way for a new era of financial sovereignty and freedom.
This work is a tribute to their endeavors and a contribution to the
continuous development and understanding of Bitcoin.

\hypertarget{foreword}{%
\section{Foreword}\label{foreword}}

The realm of Bitcoin is vast and intricate, housing within its domains
various concepts, processes, and systems that are both revolutionary and
intricate. It demands not just technical acuity but a holistic
understanding of its fundamental principles and inherent trade-offs.
Bitcoin Development Philosophy seeks to journey through this intricate
realm, unraveling the deep facets of Bitcoin, and serving as a beacon
for those who wish to delve into the complex world of Bitcoin
development. It is a guide, a companion, and a lens through which the
magnificence of Bitcoin can be appreciated in its full spectrum.

\hypertarget{preface}{%
\section{Preface}\label{preface}}

Bitcoin Development Philosophy is not merely a guide; it is a
contemplative exploration into the very essence of Bitcoin. It aims to
equip developers with a profound understanding of the philosophical
underpinnings and design paradigms of Bitcoin, enabling them to navigate
its complex realms with informed discretion and informed curiosity. This
book is the culmination of lessons learned over a decade of Bitcoin
development and public debate, designed to provide context and clarity
to both seasoned and novice developers.

\hypertarget{acknowledgements}{%
\section{Acknowledgements}\label{acknowledgements}}

Our profound gratitude to Chaincode Labs for commissioning and funding
this invaluable piece of work and to every individual who contributed
their insights and expertise in the making of this guide. We extend our
heartfelt thanks to Kalle Rosenbaum, the main author, and Linnéa
Rosenbaum, the co-author, whose invaluable insights and meticulous
efforts have shaped this book. The pursuit of knowledge is a collective
endeavor, and this work is imbued with the wisdom and experiences of
many brilliant minds from the Bitcoin development community.

\hypertarget{about-the-book}{%
\section{About the Book}\label{about-the-book}}

Bitcoin Development Philosophy is a comprehensive guide aimed at
developers who have a foundational understanding of Bitcoin concepts and
processes such as Proof-of-Work, block building, and the transaction
life cycle. This book delves into the intricacies of Bitcoin's design
trade-offs and philosophy, offering insights and context that help
developers absorb over a decade of development lessons and debates.

The book is organized into several chapters, each focusing on pivotal
topics within Bitcoin, such as decentralization, trustlessness, privacy,
finite supply, and scaling, among others. Every chapter is enriched with
links and QR codes leading to recommended articles or videos, written by
seasoned Bitcoin developers. It not only provides practical insights but
also encourages independent exploration and critical thinking about
contradictory ideas prevalent in the Bitcoin community.

\hypertarget{about-the-authors}{%
\section{About the Authors}\label{about-the-authors}}

Kalle Rosenbaum, the main author, is a seasoned software developer with
extensive experience in Bitcoin-related development since 2015. He is
the author of Grokking Bitcoin and has been a pivotal figure in
educating developers about Bitcoin. Linnéa Rosenbaum, the co-author,
holds a Ph.D. in Electronic Systems and has a rich background in
firmware and software development. She is the Swedish translator of The
Little Bitcoin Book and co-translator of The Bitcoin Standard,
contributing significantly to the dissemination of knowledge about
Bitcoin in Sweden. Their combined expertise and passion for Bitcoin have
been instrumental in creating a guide that is insightful, comprehensive,
and enlightening.

\hypertarget{decentralization}{%
\section{Decentralization}\label{decentralization}}

\includegraphics{images/decentralization-banner.jpg}

This chapter analyzes what decentralization is and why it's essential
for Bitcoin to function. We distinguish between the decentralization of
miners and that of full nodes, and discuss what they bring to the table
for censorship resistance, one of Bitcoin's most central properties. The
discussion then shifts to understanding neutrality - or
permissionlessness towards users, miners, and developers - which is a
necessary property of any decentralized system. Lastly, we touch upon
how hard it can be to grasp a decentralized system like Bitcoin, and
present some mental models that might help you grok it.

A system without any central point of control is referred to as being
\emph{decentralized}. Bitcoin is designed to avoid having a central
point of control, or more precisely a \emph{central point of
censorship}. Decentralization is a means to achieve \emph{censorship
resistance}.

There are two major aspects of decentralization in Bitcoin: miner
decentralization and full node decentralization. Miner decentralization
refers to the fact that transaction processing isn't performed nor
coordinated by any central entity. Full node decentralization refers to
the fact that validation of the blocks, i.e. the data that miners
output, gets done at the edge of the network, ultimately by its users,
and not by a few trusted authorities.

\hypertarget{minerdecentralization}{%
\subsection{Miner decentralization}\label{minerdecentralization}}

There had been attempts at creating digital currencies before Bitcoin,
but most of them failed due to a lack of governance decentralization and
censorship resistance.

Miner decentralization in Bitcoin means that the \emph{ordering of
transactions} isn't carried out by any single entity or fixed set of
entities. It's carried out collectively by all the actors who want to
participate in it; this miners' collective is a dynamic set of users.
Anyone can join or leave as they wish. This property makes Bitcoin
censorship-resistant.

If Bitcoin were centralized, it would be vulnerable to those who wished
to censor it, such as governments. It would meet the same fate as
earlier attempts to create digital money. In the introduction of
\href{https://www.blockstream.com/sidechains.pdf}{a paper} titled
``Enabling Blockchain Innovations with Pegged Sidechains'', the authors
explain how early versions of digital money weren't equipped for an
adversarial environment (see also
\protect\hyperlink{adversarialthinking}{Adversarial thinking}):

\begin{quote}
David Chaum introduced digital cash as a research topic in 1983, in a
setting with a central server that is trusted to prevent
double-spending{[}Cha83{]}. To mitigate the privacy risk to individuals
from this central trusted party, and to enforce fungibility, Chaum
introduced the blind signature, which he used to provide a cryptographic
means to prevent linking of the central server's signatures (which
represent coins), while still allowing the central server to perform
double-spend prevention. The requirement for a central server became the
Achilles' heel of digital cash{[}Gri99{]}. While it is possible to
distribute this single point of failure by replacing the central
server's signature with a threshold signature of several signers, it is
important for auditability that the signers be distinct and
identifiable. This still leaves the system vulnerable to failure, since
each signer can fail, or be made to fail, one by one.

---  various authors Enabling Blockchain Innovations with Pegged
Sidechains (2014)
\end{quote}

It became clear that using a central server to order transactions was
not a viable option due to the high risk of censorship. Even if one
replaced the central server with a federation of a fixed set of n
servers, of which at least m must approve of an ordering, there would
still be difficulties. The problem would indeed shift to one where users
must agree on this set of n servers as well as on how to replace
malicious servers with good ones without relying on a central authority.

Let's contemplate what could happen if Bitcoin were censorable. The
censor could pressure users to identify themselves, to declare where
their money is coming from or what they're buying with it before
allowing their transactions to enter the blockchain.

Also, the lack of censorship resistance would allow the censor to coerce
users into adopting new system rules. For example, they could impose a
change that allowed them to inflate the money supply, thereby enriching
themselves. In such an event, a user verifying blocks would have three
options to handle the new rules:

\begin{itemize}
\item
  Adopt: Accept the changes and adopt them into their full node.
\item
  Reject: Refuse to adopt the changes; this leaves the user with a
  system that doesn't process transactions anymore, as the censor's
  blocks are now deemed invalid by the user's full node.
\item
  Move: Appoint a new central point of control; all of the users must
  figure out how to coordinate and then agree on the new central control
  point. If they succeed, the same issues will most likely resurface at
  some point in the future, considering that the system remained just as
  censorable as it was before.
\end{itemize}

None of these options are beneficial to the user.

Censorship resistance through decentralization is what separates Bitcoin
from other money systems, but it is not an easy thing to accomplish due
to the \emph{double-spending problem}. This is the problem of making
sure no one can spend the same coin twice, an issue that many people
thought was impossible to solve in a decentralized fashion. Satoshi
Nakamoto write in his \href{https://bitcoin.org/bitcoin.pdf}{Bitcoin
whitepaper} about how to solve the double-spending problem:

\begin{quote}
In this paper, we propose a solution to the double-spending problem
using a peer-to-peer distributed timestamp server to generate
computational proof of the chronological order of transactions.

---  Satoshi Nakamoto Bitcoin: A Peer-to-Peer Electronic Cash System
(2008)
\end{quote}

Here he uses the peculiar-sounding phrase ``peer-to-peer distributed
timestamp server''. The keyword here is \emph{distributed}, which in
this context means that there is no central point of control. Nakamoto
then goes on to explain how proof-of-work is the solution. Still, no one
explains it better than
\href{https://www.reddit.com/r/Bitcoin/comments/ddddfl/question_on_the_vulnerability_of_bitcoin/f2g9e7b/}{Gregory
Maxwell on Reddit}, where he responds to someone who proposes to limit
miners' hash power to avoid potential 51\% attacks:

\begin{quote}
A decentralized system like Bitcoin uses a public election. But you
can't just have a vote of 'people' in a decentralized system because
that would require a centralized party to authorize people to vote.
Instead, Bitcoin uses a vote of computing power because it's possible to
verify computing power without the help of any centralized third party.

---  Gregory Maxwell r/Bitcoin subreddit (2019)
\end{quote}

The post explains how the decentralized Bitcoin network can come to an
agreement on transaction ordering through the use of proof-of-work. He
then concludes by saying that the 51\% attack is not particularly
worrisome, compared to people not caring about or not understanding
Bitcoin's decentralization properties.

\begin{quote}
A far bigger risk to Bitcoin is that the public using it won't
understand, won't care, and won't protect the decentralization
properties that make it valuable over centralized alternatives in the
first place.

---  Gregory Maxwell r/Bitcoin subreddit (2019)
\end{quote}

The conclusion is an important one. If people don't protect Bitcoin's
decentralization, which is a proxy for its censorship resistance,
Bitcoin might fall victim to centralizing powers, until it's so
centralized that censorship becomes a thing. Then most, if not all, of
its value proposition is gone. This brings us to the next section on
full node decentralization.

\hypertarget{fullnodedecentralization}{%
\subsection{Full node decentralization}\label{fullnodedecentralization}}

In the paragraphs above, we've mostly talked about miner
decentralization and how centralizating miners can allow for censorship.
But there's also another aspect of decentralization, namely \emph{full
node decentralization}.

The importance of full node decentralization is related to trustlessness
(see \protect\hyperlink{trustlessness}{Trustlessness}). Suppose a user
stops running their own full node due to, for example, a prohibitive
increase in the cost of operation. In that case, they have to interact
with the Bitcoin network in some other way, possibly by using web
wallets or lightweight wallets, which requires a certain level of trust
in the providers of these services. The user goes from directly
enforcing the network consensus rules to trusting that someone else
will. Now suppose that most users delegate consensus enforcement to a
trusted entity. In that case, the network can quickly spiral into
centralization, and the network rules can be changed by conspiring
malicious actors.

In
\href{https://bitcoinmagazine.com/technical/decentralist-perspective-bitcoin-might-need-small-blocks-1442090446}{a
Bitcoin Magazine article}, Aaron van Wirdum interviews Bitcoin
developers about their views on decentralization and the risks involved
in increasing Bitcoin's maximum block size. This discussion was a hot
topic during the 2014-2017 era, when many people argued over increasing
the block size limit to allow for more transaction throughput.

A powerful argument against increasing the block size is that it
increases the cost of verification (see
\protect\hyperlink{verticalscaling}{the Scaling chapter}). If
verification cost rises, it will push some users to stop running their
full nodes. This, in turn, will lead to more people not being able to
use the system in a trustless way. Pieter Wuille is quoted in the
article, where he explains the risks of full node centralization.

\begin{quote}
If lots companies run a full node, it means they all need to be
convinced to implement a different rule set. In other words: the
decentralization of block validation is what gives consensus rules their
weight. But if full node count would drop very low, for instance because
everyone uses the same web-wallets, exchanges and SPV or mobile wallets,
regulation could become a reality. And if authorities can regulate the
consensus rules, it means they can change anything that makes Bitcoin
Bitcoin. Even the 21 million bitcoin limit.

---  Pieter Wuille The Decentralist Perspective or Why Bitcoin Might
Need Small Blocks (2015)
\end{quote}

There you go. Bitcoin users should run their own full nodes to deter
regulators and big corporations from trying to change the consensus
rules.

\hypertarget{neutrality}{%
\subsection{Neutrality}\label{neutrality}}

Bitcoin is neutral, or permissionless, as people like to call it. This
means that Bitcoin doesn't care who you are or what you use it for.

\begin{quote}
bitcoin is neutral, which is a good thing, and the only way it can work.
if it was controlled by an organisation it'd just be another virtual
object type and I would have zero interest in it

---  wumpus on freenode IRC (punctuation added) \#bitcoin-core-dev
2012-04-04T17:34:04 UTC
\end{quote}

As long as you play by the rules, you're free to use it as you please,
without asking anyone for permission. This includes \emph{mining},
\emph{transacting} in, and \emph{building protocols and services} on top
of Bitcoin.

\begin{itemize}
\item
  If \textbf{mining} were a permissioned process, we would need a
  central authority to select who's allowed to mine. This would most
  likely lead to miners having to sign legal contracts in which they
  would agree to censor transactions according to the whims of the
  central authority, which defeats the purpose of mining in the first
  place.
\item
  If people \textbf{transacting} in Bitcoin had to provide personal
  information, declare what their transactions were for, or otherwise
  prove that they were worthy of transacting, we would also need a
  central point of authority to approve users or transactions. Again,
  this would lead to censorship and exclusion.
\item
  If developers had to ask for permission to \textbf{build protocols} on
  top of Bitcoin, only the protocols allowed by the central developer
  granting committee would get developed. This would, due to government
  intervention, inevitably exclude all privacy-preserving protocols and
  all attempts at improving decentralization.
\end{itemize}

At all levels, trying to impose restrictions on who gets to use Bitcoin
for what will hurt Bitcoin to the point where it's no longer living up
to its value proposition.

Pieter Wuille
\href{https://bitcoin.stackexchange.com/a/92055/69518}{answers a
question on Stack Exchange} about how the blockchain relates to normal
databases. He explains how permissionlessness is achievable through the
use of proof-of-work in combination with economic incentives. He
concludes:

\begin{quote}
Using trustless consensus algorithms like PoW does add something no
other construction gives you (permissionless participation, meaning
there is no set group of participants that can censor your changes), but
comes at a high cost, and its economic assumptions make it pretty much
only useful for systems that define their own cryptocurrency. There is
probably only place in the world for one or a few actually used ones of
these.

---  Pieter Wuille Stack Exchange (2019)
\end{quote}

He explains that, in order to achieve permissionlessness, the system
most likely needs its own currency, thereby ``limiting the use cases to
effectively just cryptocurrencies''. This is because permissionless
participation, or mining, requires economic incentives built into the
system itself.

\hypertarget{_grokking_decentralization}{%
\subsection{Grokking
decentralization}\label{_grokking_decentralization}}

A compelling aspect of Bitcoin is how hard it is to grasp that no one
controls it. There are no committees or executives in Bitcoin. Gregory
Maxwell, again
\href{https://www.reddit.com/r/Bitcoin/comments/s82t2n/comment/htdte7w/?utm_source=share\&utm_medium=web2x\&context=3}{on
the Bitcoin subreddit}, compares this to the English language in an
intriguing way:

\begin{quote}
Many people have a hard time understanding autonomous systems, there are
many in their lives things like the english language-\/- but people just
take them for granted and don't even think of them as systems. They're
stuck in a centralized way of thinking where everything they think of as
a 'thing' has an authority that controls it.

Bitcoin doesn't focus on anything. Various people who have adopted
Bitcoin chose of their own free will to promote it, and how they choose
to do so is their own business. Authority fixated people may see these
activities and believe they're some operation by the bitcoin authority,
but no such authority exists.

---  Gregory Maxwell r/Bitcoin subreddit (2022)
\end{quote}

\begin{figure}
\centering
\includegraphics{images/fishschool.jpg}
\caption{Fish schools have no leaders.}
\end{figure}

The way Bitcoin works through decentralization resembles the
extraordinary collective intelligence found among many species in
nature. Computer scientist Radhika Nagpal speaks in a
\href{https://www.ted.com/talks/radhika_nagpal_what_intelligent_machines_can_learn_from_a_school_of_fish}{Ted
talk} about the collective behavior of fish schools and how scientists
are trying to mimic it using robots.

\begin{quote}
Secondly, and the thing that I still find most remarkable, is that we
know that there are no leaders supervising this fish school. Instead,
this incredible collective mind behavior is emerging purely from the
interactions of one fish and another. Somehow, there are these
interactions or rules of engagement between neighboring fish that make
it all work out.

---  Radhika Nagpal What intelligent machines can learn from a school of
fish (2017)
\end{quote}

She points out that many systems, either natural or artificial, can and
do work without leaders, and they are powerful and resilient. Each
individual only interacts with their immediate surroundings, but
together they form something tremendous.

No matter what you think about Bitcoin, its decentralized nature makes
it difficult to control. Bitcoin exists, and there's nothing you can do
about it. It's something to be studied, not debated.

\hypertarget{trustlessness}{%
\section{Trustlessness}\label{trustlessness}}

\includegraphics{images/trustlessness-banner.jpg}

This chapter dissects the concept of trustlessness, what it means from a
computer science perspective, and why Bitcoin has to be trustless to
retain its value proposition. We then talk about what it means to use
Bitcoin in a trustless way, and what kind of guarantees a full node can
and cannot give you. In the last section, we look at the real-world
interaction between Bitcoin and actual softwares or users, and the need
to make trade-offs between convenience and trustlessness to get anything
done at all.

People often say things like ``Bitcoin is great because it's
trustless''. What do they mean by trustless? Pieter Wuille explains this
widely used term on
\href{https://bitcoin.stackexchange.com/a/45674/69518}{Stack Exchange}:

\begin{quote}
The trust we're talking about in "trustless" is an abstract technical
term. A distributed system is called trustless when it does not require
any trusted parties to function correctly.

---  Pieter Wuille Bitcoin Stack Exchange (2016)
\end{quote}

In short, the word \emph{trustless} refers to a property of the Bitcoin
protocol whereby it can logically function without ``any trusted
parties''. This is different from the trust you inevitably have to put
into the software or hardware you run. More on this latter aspect of
trust will be discussed further in this chapter.

In centralized systems, we rely on a central actor's reputation in order
to make sure that they will take care of security or roll back in case
of issues, as well as on the legal system to sanction any violations.
These trust requirements are problematic in pseudonymous decentralized
systems - there is no possibility of recourse so there really can't be
any trust. In the introduction to
\href{https://bitcoin.org/bitcoin.pdf}{the Bitcoin whitepaper}, Satoshi
Nakamoto describes this problem:

\begin{quote}
Commerce on the Internet has come to rely almost exclusively on
financial institutions serving as trusted third parties to process
electronic payments. While the system works well enough for most
transactions, it still suffers from the inherent weaknesses of the trust
based model. Completely non-reversible transactions are not really
possible, since financial institutions cannot avoid mediating disputes.
The cost of mediation increases transaction costs, limiting the minimum
practical transaction size and cutting off the possibility for small
casual transactions, and there is a broader cost in the loss of ability
to make non-reversible payments for nonreversible services. With the
possibility of reversal, the need for trust spreads. Merchants must be
wary of their customers, hassling them for more information than they
would otherwise need. A certain percentage of fraud is accepted as
unavoidable. These costs and payment uncertainties can be avoided in
person by using physical currency, but no mechanism exists to make
payments over a communications channel without a trusted party

---  Satoshi Nakamoto Bitcoin: A Peer-to-Peer Electronic Cash System
(2008)
\end{quote}

It seems that we can't have a decentralized system based on trust, and
that's why trustlessness is important in Bitcoin.

To use Bitcoin in a trustless manner, you have to run a fully-validating
Bitcoin node. Only then will you be able to verify that the blocks you
receive from others are following the consensus rules; for example, that
the coin issuance schedule is kept and that no double-spends occur on
the blockchain. If you don't run a full node, you outsource verification
of Bitcoin blocks to someone else and trust them to tell you the truth,
which means you're not using Bitcoin trustlessly.

David Harding has authored
\href{https://bitcoin.org/en/bitcoin-core/features/validation}{an
article on the bitcoin.org website} explaining how running a full node -
or using Bitcoin trustlessly - actually helps you.

\begin{quote}
The bitcoin currency only works when people accept bitcoins in exchange
for other valuable things. That means it's the people accepting bitcoins
who give it value and who get to decide how Bitcoin should work.

When you accept bitcoins, you have the power to enforce Bitcoin's rules,
such as preventing confiscation of any person's bitcoins without access
to that person's private keys.

Unfortunately, \textbf{many users outsource their enforcement power}.
This leaves Bitcoin's decentralization in a weakened state where a
handful of miners can collude with a handful of banks and free services
to change Bitcoin's rules for all those non-verifying users who
outsourced their power.

Unlike other wallets, \textbf{Bitcoin Core does enforce the rules}---so
if the miners and banks change the rules for their non-verifying users,
those users will be unable to pay full validation Bitcoin Core users
like you.

---  David Harding Full Validation on bitcoin.org (2015)
\end{quote}

He says that running a full node will help you verify every aspect of
the blockchain without trusting anyone else, so as to ensure that the
coins you receive from others are genuine. This is great, but there's
one important thing that a full node can't help you with: it can't
prevent double- spending through chain rewrites:

\begin{quote}
Note that although all programs---including Bitcoin Core---are
vulnerable to chain rewrites, Bitcoin provides a defense mechanism: the
more confirmations your transactions have, the safer you are. There is
no known decentralized defense better than that.

---  David Harding Full Validation on bitcoin.org (2015)
\end{quote}

No matter how advanced your software is, you still have to trust that
the blocks containing your coins won't be rewritten. However, as pointed
out by Harding, you can await a number of confirmations, after which you
consider the probability of a chain rewrite small enough to be
acceptable.

The incentives for using Bitcoin in a trustless way align with the
system's need for \protect\hyperlink{fullnodedecentralization}{full node
decentralization}. The more people who use their own full nodes, the
more full node decentralization, and thus the stronger Bitcoin stands
against malicious changes to the protocol. But unfortunately, as
explained in the full node decentralization section, users often opt for
trusted services as consequence of the inevitable trade-off between
trustlessness and convenience.

Bitcoin's trustlessness is absolutely imperative from a system
perspective. In 2018, Matt Corallo,
\href{https://btctranscripts.com/baltic-honeybadger/2018/trustlessness-scalability-and-directions-in-security-models/}{spoke
about trustlessness} at the Baltic Honeybadger conference in Riga. The
essence of that talk is that you can't build trustless systems on top of
a trusted system, but you can build trusted systems - for example, a
custodial wallet - on top of a trustless system.

\begin{figure}
\centering
\includegraphics{images/trust.png}
\caption{A trustless base layer allows for various trade-offs on higher
levels.}
\end{figure}

This security model allows the system designer to select trade-offs that
make sense to them without forcing those trade-offs on others.

\hypertarget{donttrustverify}{%
\subsection{Don't trust, verify}\label{donttrustverify}}

Bitcoin works trustlessly, but you still have to trust your software and
hardware to some degree. That's because your software or hardware might
not be programmed to do what's stated on the box. For example:

\begin{itemize}
\item
  The CPU might be maliciously designed to detect private key
  cryptographic operations and leak the private key data.
\item
  The operating system's random number generator might not be as random
  as it claims.
\item
  Bitcoin Core might have sneaked in code that will send your private
  keys to some bad actor.
\end{itemize}

So, besides running a full node, you also need to make sure you're
running what you intend to. Reddit user brianddk
\href{https://www.reddit.com/r/Bitcoin/comments/smj1ep/bitcoin_v220_and_guix_stronger_defense_against/}{wrote
an article} about the various levels of trust you can choose from, when
verifying your software. In the section ``Trusting the builders'', he
talks about \emph{reproducible builds}:

\begin{quote}
Reproducible builds are a way to design software so that many community
developers can each build the software and ensure that the final
installer built is identical to what other developers produce. With a
very public, reproducible project like bitcoin, no single developer
needs to be completely trusted. Many developers can all perform the
build and attest that they produced the same file as the one the
original builder digitally signed.

---  brianddk on Reddit Bitcoin v22.0 and Guix; Stronger defense against
the "Trusting Trust Attack" (2022)
\end{quote}

The article defines 5 levels of trust: trusting the site, the builders,
the compiler, the kernel, and the hardware.

To further deepen the topic of reproducible builds, Carl Dong
\href{https://btctranscripts.com/breaking-bitcoin/2019/bitcoin-build-system/}{made
a presentation about Guix}
(\href{https://www.youtube.com/watch?v=I2iShmUTEl8}{video}) explaining
why trusting the operating system, libraries, and compilers can be
problematic, and how to fix that with a system called Guix, which is
used by Bitcoin Core today.

\begin{quote}
So what can we do about the fact that our toolchain can have a bunch of
trusted binaries that can be reproducibly malicious? We need to be more
than reproducible. We need to be bootstrappable. We cannot have that
many binary tools that we need to download and trust from external
servers controlled by other organizations. We should know how these
tools are built and exactly how we can go through the process of
building them again, preferably from a much smaller set of trusted
binaries. We need to minimize our trusted set of binaries as much as
possible, and have an easily auditable path from those toolchains to
what we use how to build bitcoin. This allows us to maximize
verification and minimize trust.

---  Carl Dong on Guix Breaking Bitcoin Conference (2019)
\end{quote}

He then explains how Guix allows us to only trust a minimal binary of
357 bytes that can be verified and fully understood if you know how to
interpret the instructions. This is quite remarkable: one verifies that
the 357-byte binary does what it should, then uses it to build the full
build system from source code, and ends up with a Bitcoin Core binary
that should be an exact copy of anyone else's build.

There's a mantra that many bitcoiners subscribe to, which captures well
much of the above:

\begin{quote}
Don't trust, verify.

---  Bitcoiners everywhere
\end{quote}

This alludes to the phrase
"\href{https://en.wikipedia.org/wiki/Trust,_but_verify}{trust, but
verify}" that former U.S. president Ronald Reagan used in the context of
nuclear disarmament.
\href{https://twitter.com/Truthcoin/status/1491415722123153408?s=20\&t=ZyROxZxlBppdRpuuzsiF5w}{Bitcoiners
switched it around to highlight the rejection of trust and the
importance of running a full node}.

It's up to the users to decide to what degree they want to verify the
software they use and the blockchain data they receive. As with so many
other things in Bitcoin, there's a trade-off between convenience and
trustlessness. It's almost always more convenient to use a custodial
wallet compared to running Bitcoin Core on your own hardware. However,
as Bitcoin software is maturing and user interfaces are improving, over
time it should get better at supporting users willing to work towards
trustlessness. Also, as users gain more knowledge over time, they should
be able to gradually remove trust from the equation.

Some users think adversarially (see
\protect\hyperlink{adversarialthinking}{Adversarial thinking}) and
verify most aspects of the software they run. As a consequence, they
reduce the need for trust to the bare minimum, as they only need to
trust their computer hardware and operating system. In doing so, they
also help people who don't verify their hardware as thoroughly by
raising their voices in public to warn about any issues they might find.
One good example of this is an
\href{https://bitcoincore.org/en/2018/09/20/notice/}{event that occurred
in 2018}, when someone discovered a bug that would allow miners to spend
an output twice in the same transaction:

\begin{quote}
CVE-2018-17144, a fix for which was released on September 18th in
Bitcoin Core versions 0.16.3 and 0.17.0rc4, includes both a Denial of
Service component and a critical inflation vulnerability. It was
originally reported to several developers working on Bitcoin Core, as
well as projects supporting other cryptocurrencies, including ABC and
Unlimited on September 17th as a Denial of Service bug only, however we
quickly determined that the issue was also an inflation vulnerability
with the same root cause and fix.

---  CVE-2018-17144 Full Disclosure on bitcoincore.org (2018)
\end{quote}

Here, an anonymous person reported an issue that turned out much worse
than the reporter realized. This highlights the fact that people who
verify the code often report security flaws instead of exploiting them.
This is beneficial to those who aren't able to verify everything
themselves. However, users should not trust others to keep them safe,
but should rather verify for themselves whenever and whatever they can;
that's how one remains as sovereign as possible, and how Bitcoin
prospers. The more eyes on the software, the less likely it is that
malicious code and security flaws slip through.

\hypertarget{privacy}{%
\section{Privacy}\label{privacy}}

\includegraphics{images/privacy-banner.jpg}

This chapter deals with how to keep your private financial information
to yourself. It explains what privacy stands for in the context of
Bitcoin, why it's important, and what it means to say that Bitcoin is
pseudonymous. It also looks into how private data can leak, both
on-chain and off-chain. Then, it talks about the fact that bitcoins
should be fungible, meaning interchangeable for any other bitcoins, and
how fungibility and privacy go hand in hand. Lastly, the chapter
introduces some measures you can take to improve your privacy and that
of others.

Bitcoin can be described as a pseudonymous system (see
\protect\hyperlink{pseudonymity}{Pseudonymity} for further details on
this), where users have multiple pseudonyms in the form of public keys.
At first glance, this looks like a pretty good way to protect users from
being identified, but it is in fact really easy to leak private
financial information unintentionally.

\hypertarget{_what_does_privacy_mean}{%
\subsection{What does privacy mean?}\label{_what_does_privacy_mean}}

Privacy can mean different things in different contexts. In Bitcoin, it
generally means that users don't have to reveal their financial
information to others, unless they voluntarily do so.

There are many ways in which you may leak your private information to
others, with or without knowing it. Data can either leak from the public
blockchain or through other means, for example when malicious actors
intercept your internet communications.

\hypertarget{whyprivacyimportant}{%
\subsection{Why is privacy important?}\label{whyprivacyimportant}}

It may seem obvious why privacy is important in Bitcoin, but there are
some aspects of it that one might not immediately think about.
\href{https://bitcointalk.org/index.php?topic=334316.msg3588908\#msg3588908}{On
the Bitcoin Talk forum}, Gregory Maxwell walks us through a lot of good
reasons why he thinks privacy matters. Among them are free market,
safety, and human dignity:

\begin{quote}
Financial privacy is an essential criteria for the efficient operation
of a free market: if you run a business, you cannot effectively set
prices if your suppliers and customers can see all your transactions
against your will. You cannot compete effectively if your competition is
tracking your sales. Individually your informational leverage is lost in
your private dealings if you don't have privacy over your accounts: if
you pay your landlord in Bitcoin without enough privacy in place, your
landlord will see when you've received a pay raise and can hit you up
for more rent.

Financial privacy is essential for personal safety: if thieves can see
your spending, income, and holdings, they can use that information to
target and exploit you. Without privacy malicious parties have more
ability to steal your identity, snatch your large purchases off your
doorstep, or impersonate businesses you transact with towards
you\ldots\hspace{0pt} they can tell exactly how much to try to scam you
for.

Financial privacy is essential for human dignity: no one wants the
snotty barista at the coffee shop or their nosy neighbors commenting on
their income or spending habits. No one wants their baby-crazy in-laws
asking why they're buying contraception (or sex toys). Your employer has
no business knowing what church you donate to. Only in a perfectly
enlightened discrimination free world where no one has undue authority
over anyone else could we retain our dignity and make our lawful
transactions freely without self-censorship if we don't have privacy.

---  Gregory Maxwell Bitcoin Talk forum (2013)
\end{quote}

Maxwell also touches on fungibility, which will be discussed
\protect\hyperlink{fungibility}{later in this chapter}, as well as on
how privacy and law enforcement are not contradictory.

\hypertarget{pseudonymity}{%
\subsection{Pseudonymity}\label{pseudonymity}}

We mentioned above that Bitcoin is pseudonymous, and that the pseudonyms
are public keys. In the media you often hear that Bitcoin is anonymous,
which is not correct. There is a distinction between anonymity and
pseudonymity.

Andrew Poelstra
\href{https://bitcoin.stackexchange.com/a/29473/69518}{explains in a
Bitcoin Stack Exchange post} what anonymity would look like in
transactions:

\begin{quote}
Total anonymity, in the sense that when you spend money there is no
trace of where it came from or where it's going, is theoretically
possible by using the cryptographic technique of zero-knowledge proofs.

---  Andrew Poelstra on anonymity Bitcoin Stack Exchange (2016)
\end{quote}

The difference seems to be that in a pseudonymous form of money you can
trace payments between pseudonyms, whereas in an anonymous form of money
you can't. Since bitcoin payments are traceable between pseudonyms, it's
not an anonymous system.

We have also said that the pseudonyms are public keys, but it's actually
addresses derived from public keys. Why do we use addresses as
pseudonyms and not something else, for example some descriptive names,
like ``watchme1984''? This has been
\href{https://bitcoin.stackexchange.com/a/25175/69518}{explained well}
by user Tim S., also on Bitcoin Stack Exchange:

\begin{quote}
In order for Bitcoin's idea to work, you must have coins that can only
be spent by the owner of a given private key. This means that whatever
you send to must be tied, in some way, to a public key.

Using arbitrary pseudonyms (e.g. user names) would mean that you'd have
to then somehow link the pseudonym to a public key in order to enable
public/private key crypto. This would remove the ability to securely
create addresses/pseudonyms offline (e.g. before someone could send
money to the user name "tdumidu", you'd have to announce in the
blockchain that "tdumidu" is owned by public key
"a1c\ldots\hspace{0pt}", and include a fee so others have a reason to
announce it), reduce anonymity (by encouraging you to reuse pseudonyms),
and needlessly bloat the size of the blockchain. It would also create a
false sense of security that you're sending to who you think you are (if
I take the name "Linus Torvalds" before he does, then it's mine and
people might send money thinking they're paying the creator of Linux,
not me).

---  Tim S. on pseudonyms Bitcoin Stack Exchange (2014)
\end{quote}

By using addresses, or public keys, we achieve important goals, such as
removing the need to somehow register a pseudonym beforehand, reducing
the incentives for pseudonym reuse, avoiding blockchain bloat, and
making it harder to impersonate other people.

\hypertarget{blockchainprivacy}{%
\subsection{Blockchain privacy}\label{blockchainprivacy}}

Blockchain privacy refers to the information you disclose by transacting
on the blockchain. It applies to all transactions, the ones you send as
well as the ones you receive.

Satoshi Nakamoto ponders over on-chain privacy in section 7 of his
\href{https://bitcoin.org/bitcoin.pdf}{Bitcoin whitepaper}:

\begin{quote}
As an additional firewall, a new key pair should be used for each
transaction to keep them from being linked to a common owner. Some
linking is still unavoidable with multi-input transactions, which
necessarily reveal that their inputs were owned by the same owner. The
risk is that if the owner of a key is revealed, linking could reveal
other transactions that belonged to the same owner.

---  Satoshi Nakamoto Bitcoin: A Peer-to-Peer Electronic Cash System
(2008)
\end{quote}

The paper summarizes the main problems of blockchain privacy, namely
address reuse and address clustering. The first is self-explaining, the
latter refers to the ability to decide, with some level of certainty,
that a set of different addresses belongs to the same user.

\begin{figure}
\centering
\includegraphics{images/address-reuse-clustering.png}
\caption{Typical privacy leaks on the blockchain.}
\end{figure}

Chris Belcher
\href{https://en.bitcoin.it/Privacy\#Blockchain_attacks_on_privacy}{wrote
in great detail} about the different kinds of privacy leaks that can
happen on the Bitcoin blockchain. We recommend you read at least the
first few subsections under ``Blockchain attacks on privacy.''

The takeaway is that privacy in Bitcoin isn't perfect. It requires a
significant amount of work to transact privately. Most people aren't
prepared to go that far for privacy. There seems to be a clear trade-off
between privacy and usability.

Another important aspect of privacy is that the measures you take to
protect your own privacy affect other users as well. If you are sloppy
with your own privacy, other people might experience reduced privacy,
too. Gregory Maxwell explains this very plainly on the same Bitcoin Talk
discussion
\href{https://bitcointalk.org/index.php?topic=334316.msg3589252\#msg3589252}{that
we linked above}, and concludes with an example:

\begin{quote}
This actually works in practice, too\ldots\hspace{0pt} A nice whitehat
hacker on IRC was playing around with brainwallet cracking and hit a
phrase with \textasciitilde250 BTC in it. We were able to identify the
owner from just the address alone, because they'd been paid by a Bitcoin
service that reused addresses and he was able to talk them into giving
up the users contact information. He actually got the user on the phone,
they were shocked and confused--- but grateful to not be out their coin.
A happy ending there. (This isn't the only example of it, by far
\ldots\hspace{0pt} but its one of the more fun ones).

---  Gregory Maxwell Bitcoin Talk forum (2013)
\end{quote}

In this case, it all went well thanks to the philanthropically-minded
hacker, but don't count on that next time.

\hypertarget{nonblockchainprivacy}{%
\subsection{Non-blockchain privacy}\label{nonblockchainprivacy}}

While the blockchain proves to be a notorious source of privacy leaks,
there are plenty of other leaks that don't use the blockchain, some
sneakier than others. These range from key-loggers to network traffic
analysis. To read up on some of these methods, please refer again to
\href{https://en.bitcoin.it/Privacy\#Non-blockchain_attacks_on_privacy}{Chris
Belcher's piece}, specifically the section ``Non-blockchain attacks on
privacy''.

Among a plethora of attacks, Belcher mentions the possibility of someone
snooping on your internet connection, for example, your ISP:

\begin{quote}
If the adversary sees a transaction or block coming out of your node
which did not previously enter, then it can know with near-certainty
that the transaction was made by you or the block was mined by you. As
internet connections are involved, the adversary will be able to link
the IP address with the discovered bitcoin information.

---  Chris Belcher Bitcoin wiki
\end{quote}

However, among the most obvious privacy leaks are exchanges. Due to
laws, usually referred to as KYC (Know Your Customer) and AML
(Anti-Money Laundering), that are valid in the jurisdictions they
operate in, exchanges and related companies often have to collect
personal data about their users, building up big databases about which
users own which bitcoins. These databases are great honeypots for evil
governments and criminals who are always on the lookout for new victims.
There are actual markets for this kind of data, where hackers sell data
to the highest bidder. To make things worse, the companies that manage
these databases often have little experience with protecting financial
data, in fact many of them are young start-ups, and we know for a fact
that several leaks have already occurred. A few examples are
\href{https://bitcoinmagazine.com/business/probably-the-largest-kyc-data-leak-in-history-demonstrates-the-importance-of-bitcoin-privacy}{India-based
MobiQwik} and
\href{https://bitcoinmagazine.com/business/hubspot-security-breach-leaks-bitcoin-users-data}{HubSpot}

Again, protecting data against this wide range of attacks is hard, and
it is likely that you won't be fully able to do so. You'll have to opt
for the trade-off between convenience and privacy that works best for
you.

\hypertarget{fungibility}{%
\subsection{Fungibility}\label{fungibility}}

Fungibility, in the context of currencies, means that one coin is
interchangeable for any other coin of the same currency. This funny word
was briefly touched upon in \protect\hyperlink{whyprivacyimportant}{Why
is privacy important?}. In the article discussed there, Gregory Maxwell
\href{https://bitcointalk.org/index.php?topic=334316.msg3588908\#msg3588908}{stated}

\begin{quote}
Financial privacy is an essential element to fungibility in Bitcoin: if
you can meaningfully distinguish one coin from another, then their
fungibility is weak. If our fungibility is too weak in practice, then we
cannot be decentralized: if someone important announces a list of stolen
coins they won't accept coins derived from, you must carefully check
coins you accept against that list and return the ones that fail.
Everyone gets stuck checking blacklists issued by various authorities
because in that world we'd all not like to get stuck with bad coins.
This adds friction and transactional costs and makes Bitcoin less
valuable as a money.

---  Gregory Maxwell Bitcoin Talk forum (2013)
\end{quote}

Here, he speaks about the dangers derived from a lack of fungibility.
Suppose that you have a UTXO. That UTXO's history can normally be traced
back several hops, fanning out to multitudes of previous outputs. If any
of those outputs were involved in any illegal, unwanted, or suspicious
activity, then some potential recipients of your coin might reject it.
If you think that your payees will verify your coins against some
centralized whitelist or blacklist service, you might start checking the
coins you receive too, just to be on the safe side. The result is that
bad fungibility will bolster even worse fungibility.

Adam Back and Matt Corallo
\href{https://btctranscripts.com/scalingbitcoin/milan-2016/fungibility-overview/}{gave
a presentation about fungibility} at Scaling Bitcoin in Milan in 2016.
They were thinking along the same lines:

\begin{quote}
You need fungibility for bitcoin to function. If you receive coins and
can't spend them, then you start to doubt whether you can spend them. If
there are doubts about coins you receive, then people are going to go to
taint services and check whether ``are these coins blessed'' and then
people are going to refuse to trade. What this does is it transitions
bitcoin from a decentralized permissionless system into a centralized
permissioned system where you have an ``IOU'' from the blacklist
providers.

---  Matt Corallo and Adam Back Fungibility Overview (2016)
\end{quote}

It seems that privacy and fungibility go hand-in-hand. Fungibility will
weaken if privacy is weak, for example as coins from unwanted people may
become blacklisted. In the same way, privacy will weaken if fungibility
is weak: if there is a blacklist, you will have to ask the blacklist
providers about which coins to accept, thereby possibly revealing your
IP address, email address, and other sensitive information. These two
features are so intertwined that it's hard to talk about either of them
in isolation.

\hypertarget{privacymeasures}{%
\subsection{Privacy measures}\label{privacymeasures}}

Several techniques have been developed to help people protecting
themselves from privacy leaks. Among the most obvious ones is, as noted
by Nakamoto\textgreater\textgreater{} in
\protect\hyperlink{blockchainprivacy}{Blockchain privacy}, using unique
addresses for every transaction, but several others exist. We're not
going to teach you how to become a privacy ninja. However, Bitcoin Q+A
has a quick summary of privacy-enhancing technologies, somewhat ordered
by how hard they are to implement, at
\url{https://bitcoiner.guide/privacytips/}. When you read it, you'll
notice that Bitcoin privacy often has to do with stuff outside of
Bitcoin. For example, you shouldn't brag about your bitcoins, and you
should use Tor and VPN. The post also lists some measures directly
related to Bitcoin:

\begin{description}
\item[Full node]
If you don't use your own full node, you will leak lots of information
about your wallet to servers on the internet. Running a full node is a
great first step.
\item[Lightning Network]
Several protocols exist on top of Bitcoin, for example the Lightning
Network and Blockstream's Liquid sidechain.
\item[CoinJoin]
A way for multiple people to merge their transactions into one, making
it harder to do chain analysis.
\end{description}

In
\href{https://btctranscripts.com/breaking-bitcoin/2019/breaking-bitcoin-privacy/}{a
talk} at the Breaking Bitcoin conference, Chris Belcher gave an
interesting practical example of how privacy has been improved.

\begin{quote}
They were a bitcoin casino. Online gambling is not allowed in the US.
Any customers of Coinbase that deposited straight to Bustabit would have
their accounts shutdown because Coinbase was monitoring for this.
Bustabit did a few things. They did something called change avoidance
where you go through-- and you see if you can construct a transaction
that has no change output. This saves miner fees and also hinders
analysis. Also, they imported their heavily-used reused deposit
addresses into joinmarket. At this point, coinbase.com customers never
got banned. It seems Coinbase's surveillance service was unable to do
the analysis after this, so it is possible to break these algorithms.

---  Chris Belcher in "Breaking Bitcoin Privacy" Breaking Bitcoin
conference (2019)
\end{quote}

He also mentioned this example, among others, on the
\href{https://en.bitcoin.it/Privacy}{Privacy page} on the Bitcoin wiki.

Note how better privacy can be achieved by building systems on top of
Bitcoin, as is the case with Lightning Network:

\begin{figure}
\centering
\includegraphics{images/privacy.png}
\caption{Layers on top of Bitcoin can add privacy.}
\end{figure}

We noted in \protect\hyperlink{trustlessness}{Trustlessness} that the
need for trust can only increase with layers on top, but that doesn't
seem to be the case for privacy, which can be improved or made worse
arbitrarily in layers on top. Why is that? Any layer on top of Bitcoin,
as explained in \protect\hyperlink{layeredscaling}{Layered scaling},
must use on-chain transactions occasionally, otherwise it wouldn't be
``on top of Bitcoin''. Privacy-enhancing layers generally try to use the
base layer as little as possible to minimize the amount of information
revealed.

The above are somewhat technical ways to improve your privacy. But there
are other ways. At the beginning of this chapter, we said that Bitcoin
is a pseudonymous system. This means that users in Bitcoin aren't known
by their real names or other personal data, but by their public keys. A
public key is a pseudonym for a user, and a user can have multiple
pseudonyms. In an ideal world, your in-person identity is decoupled from
your Bitcoin pseudonyms. Unfortunately, due to the privacy problems
described in this chapter, this decoupling usually degrades over time.

To mitigate the risks of having your personal data revealed is to not
give it out in the first place nor to give it to centralized services,
which build big databases that can leak (see
\protect\hyperlink{nonblockchainprivacy}{Non-blockchain privacy}). An
article by Bitcoin Q+A
\href{https://bitcoiner.guide/nokyconly/}{explains KYC} and the dangers
derived from it. It also suggests some steps you can take to improve
your situation.

\begin{quote}
Thankfully there are some options out there to purchase Bitcoin via no
KYC sources. These are all P2P (peer to peer) exchanges where you are
trading directly with another individual and not a centralised third
party. Unfortunately some sell other coins as well as bitcoin so we urge
you to take care.

---  Bitcoin Q+A, noKYC only, Avoid the creep bitcoiner.guide
\end{quote}

The article suggests you avoid using exchanges that require KYC/AML and
instead trade in private, or use decentralized exchanges like
\href{https://bisq.network/}{bisq}.

For more in-depth reading about countermeasures, refer to the previously
mentioned
\href{https://en.bitcoin.it/wiki/Privacy\#Methods_for_improving_privacy_.28non-blockchain.29}{wiki
article on privacy}, starting at ``Methods for improving privacy
(non-blockchain)''.

\hypertarget{_conclusion}{%
\subsection{Conclusion}\label{_conclusion}}

Privacy is very important but hard to achieve. There is no privacy
silver bullet. To get decent privacy in Bitcoin, you have to take active
measures, some of which are costly and time-consuming.

\hypertarget{finitesupply}{%
\section{Finite supply}\label{finitesupply}}

\includegraphics{images/finitesupply-banner.jpg}

This chapter looks into the bitcoin supply limit of 21 million BTC, or
how much is it actually? We talk about how this limit is enforced and
what one can do to verify that it's being respected. Moreover, we take a
peek into the crystal ball and discuss the dynamics that will come into
play when the block reward shifts from subsidy-based to fee-based.

The well-known finite supply of 21 million BTC is regarded as a
fundamental property of Bitcoin. But is it really set in stone?

Let's start by looking at what the current consensus rules say about the
supply of bitcoin, and how much of it will actually be usable. Pieter
Wuille wrote a piece about this
\href{https://bitcoin.stackexchange.com/a/38998/69518}{on Stack
Exchange}, in which he counted how many bitcoins there would be once all
coins are mined:

\begin{quote}
If you sum all these numbers together, you get \textbf{20999999.9769}
BTC.

---  Pieter Wuille Stack Exchange (2015)
\end{quote}

But due to a number of reasons --- such as early problems with coinbase
transactions, miners who unintentionally claim less than allowed, and
loss of private keys --- that upper limit will never be reached. Wuille
concludes:

\begin{quote}
This leaves us with \textbf{20999817.31308491} BTC (taking everything up
to block 528333 into account)

... However, various wallets have been lost or stolen, transactions have
been sent to the wrong address, people forgot they owned bitcoin. The
totals of this may well be millions. People have tried to tally known
losses up \href{https://bitcointalk.org/index.php?topic=7253.0}{here}.

This leaves us with: \textbf{???} BTC.

---  Pieter Wuille Stack Exchange (2015)
\end{quote}

We can thus be sure that the bitcoin supply will be 20999817.31308491
BTC at most. Any lost or unverifiably burnt coins will make this number
lower, but we don't know by how much. The interesting thing is that it
doesn't really matter, or better yet it does matter in a positive way
for bitcoin holders,
\href{https://bitcointalk.org/index.php?topic=198.msg1647\#msg1647}{as
explained} by Satoshi Nakamoto:

\begin{quote}
Lost coins only make everyone else's coins worth slightly more. Think of
it as a donation to everyone.

---  Satoshi Nakamoto on lost bitcoins Bitcointalk forum (2010)
\end{quote}

The finite supply will shrink and this should, at least in theory, cause
price deflation.

More important than the exact number of coins in circulation is the way
the supply limit is enforced without any central authority. Alias
chytrik puts it well on
\href{https://bitcoin.stackexchange.com/a/106830/69518}{Stack Exchange}.

\begin{quote}
So the answer is that you don't have to trust someone to not increase
the supply. You just have to run some code that will verify that they
haven't.

---  chytrik Stack Exchange (2021)
\end{quote}

Even if some full nodes turn to the dark side and decide to accept
blocks with higher-value coinbase transactions, all the remaining full
nodes will simply neglect them and continue doing business as usual.
Some full nodes may, intentionally or unintentionally (see
\protect\hyperlink{combined-output-overflow}{2010-08-15 Combined output
overflow (CVE-2010-5139)}), run evil softwares, yet the collective will
robustly secure the blockchain. In conclusion, you can choose to trust
the system without having to trust anyone.

\hypertarget{_block_subsidy_and_transaction_fees}{%
\subsection{Block subsidy and transaction
fees}\label{_block_subsidy_and_transaction_fees}}

A block reward is composed of the block subsidy plus transaction fees.
The block reward needs to cover Bitcoin's security costs. We can say for
sure that under today's conditions with regard to block subsidy,
transaction fees, bitcoin price, mempool size, hash power, degree of
decentralization etc., the incentives for every player to play by the
rules are high enough to preserve a secure monetary system.

What happens when the block subsidy approaches zero? To keep things
simple, let's assume it actually equals zero. At this point, the
system's security cost is covered through transaction fees only. What
the future holds for us when this happens, we cannot know. The
uncertainty factors are numerous and we are left to speculations. For
example, Paul Sztorc's contribution to the subject
\href{https://www.truthcoin.info/blog/security-budget/}{in his Truthcoin
blog} is mostly speculations, but he has at least one solid point
(please note that M2, as referred to by Sztorc, is a measurement of a
fiat money supply):

\begin{quote}
While the two are mixed into the same ``security budget'', the block
subsidy and txn-fees are utterly and completely different. They are as
different from each other, as ``VISA's total profits in 2017'' are from
the ``total increase in M2 in 2017''.

---  Paul Sztorc, Security Budget in the Long Run Truthcoin blog (2019)
\end{quote}

Today, it is holders who pay for security (via monetary inflation).
Tomorrow it will be the spenders' turn to somehow shoulder this burden,
as illustrated below.

\begin{figure}
\centering
\includegraphics{images/finitesupply.png}
\caption{As time goes by, the bearing of security costs will shift from
holders to spenders.}
\end{figure}

When transaction fees are the main motivation for mining, the incentives
shift. Most notably, if the mempool of a miner doesn't contain enough
transaction fees, it might become more profitable for that miner to
rewrite Bitcoin's history rather than extending it. Bitcoin Optech has a
specific \href{https://bitcoinops.org/en/topics/fee-sniping/}{section on
this behavior}, called \emph{fee sniping}, written by David Harding:

\begin{quote}
Fee sniping is a problem that may occur as Bitcoin's subsidy continues
to diminish and transaction fees begin to dominate Bitcoin's block
rewards. If transaction fees are all that matter, then a miner with
\texttt{x} percent of the hash rate has a \texttt{x} percent chance of
mining the next block, so the expected value to them of honestly mining
is \texttt{x} percent of the
\href{https://bitcoinops.org/en/newsletters/2021/06/02/\#candidate-set-based-csb-block-template-construction}{best
feerate set of transactions} in their mempool.

Alternatively, a miner could dishonestly attempt to re-mine the previous
block plus a wholly new block to extend the chain. This behavior is
referred to as fee sniping, and the dishonest miner's chance of
succeeding at it if every other miner is honest is
\texttt{(x/(1-x))\^{}2}. Even though fee sniping has an overall lower
probability of success than honest mining, attempting dishonest mining
could be the more profitable choice if transactions in the previous
block paid significantly higher feerates than the transactions currently
in the mempool---a small chance at a large amount can be worth more than
a large chance at a small amount.

---  David Harding, fee sniping Bitcoin Optech website
\end{quote}

Throwing a wet blanket over our hopes for the future is the fact that if
miners start conducting fee sniping, this will incentivize others to do
the same, leaving even fewer honest miners. This could severely impair
the overall security of Bitcoin. Harding goes on to list a few
countermeasures that can be taken, such as relying on transaction time
locks to restrict where in the blockchain the transaction may appear.

So, given that the consensus on finite supply remains, the block subsidy
will - thanks to
\href{https://github.com/bitcoin/bips/blob/master/bip-0042.mediawiki}{BIP42}
which fixed a very-long-term inflation bug - get to zero around year
2140. Will the transaction fees thereafter be enough to secure the
network? It's impossible to say, but we do know a few things:

\begin{itemize}
\item
  A century is a \emph{long} time from the Bitcoin perspective. If it is
  still around, it will have probably evolved enormously.
\item
  If an overwhelming economic majority finds it necessary to change the
  rules and introduce for example a perpetual annual 0.1\% or 1\%
  monetary inflation, the supply of bitcoin will no longer be finite.
\item
  With zero block subsidy and an empty or nearly empty mempool, things
  can become shaky due to fee sniping.
\end{itemize}

Since the transition to a fee-only block reward is so far in the future,
it might be wise not to jump to conclusions and try to fix the potential
issues while we can. For example, Peter Todd thinks there's an actual
risk that Bitcoin's security budget won't be enough in the future, and
consequently argues for a small perpetual inflation in Bitcoin. However,
he also thinks it's not a good idea to discuss such an issue at this
time, as
\href{https://www.whatbitcoindid.com/podcast/peter-todd-on-the-essence-of-bitcoin}{he
said on the What Bitcoin Did podcast}:

\begin{quote}
But, that's a risk like 10, 20 years in the future. That is a very long
time. And, by then, who the hell knows what the risks are?

---  Peter Todd on security budget What Bitcoin Did podcast (2019)
\end{quote}

Perhaps we could think of Bitcoin as something organic. Imagine a small,
slowly-growing oak plant. Imagine also that you have never seen a fully
grown tree in your life. Wouldn't it be wise then to restrain your
control issues instead of setting in advance all the rules on how this
plant should be allowed to evolve and grow?

\hypertarget{_conclusion_2}{%
\subsection{Conclusion}\label{_conclusion_2}}

Whether the bitcoin supply will grow past 21 million we cannot say
today, and that is probably not so bad. Ensuring that the security
budget remains high enough is crucial but not urgent. Let's have this
discussion in 10-50 years, when we know more. If it's still relevant.

\hypertarget{upgrading}{%
\section{Upgrading}\label{upgrading}}

\includegraphics{images/upgrading-banner.jpg}

Upgrading Bitcoin in a safe way can be extremely difficult. Some changes
take several years to roll out. In this chapter, we learn about the
common vocabulary around upgrading Bitcoin, and explore some examples of
historic upgrades to its protocol as well as the insights that we gained
from them. Finally, we talk about chain splits and the risks and costs
related to them.

To get in tune for this chapter, you should read
\href{https://bitcointalk.org/dec/p1.html}{David Harding's piece on
harmony and discord}.

\begin{quote}
Bitcoin experts talk often of consensus, whose meaning is abstract and
hard to pin down. But the word consensus evolved from the Latin word
concentus, "a singing together,
harmony,"{[}\href{https://bitcointalk.org/dec/p1.html\#ftnt1}{1}{]} so
let us talk not of Bitcoin consensus but of Bitcoin harmony.

Harmony is what makes Bitcoin work. Thousands of full nodes each work
independently to verify the transactions they receive are valid,
producing a harmonious agreement about the state of the Bitcoin ledger
without any node operator needing to trust anyone else. It's similar to
a chorus where each member sings the same song at the same time to
produce something far more beautiful than any of them could produce
alone.

The result of Bitcoin harmony is a system where bitcoins are safe not
just from petty thieves (provided you keep your keys secure) but also
from endless inflation, mass or targeted confiscation, or simply the
bureaucratic morass that is the legacy financial system.

---  David Harding Harmony and Discord
\end{quote}

This chapter discusses how Bitcoin can be upgraded without causing
discord. Staying in harmony, i.e. maintaining consensus, is indeed one
of the biggest challenges in Bitcoin development. There are lots of
nuances to upgrade mechanisms, which might be best understood by
studying actual cases of previous upgrades. For this reason, the chapter
puts much focus on historic examples, and it starts by setting the stage
with some useful vocabulary.

\hypertarget{_vocabulary}{%
\subsection{Vocabulary}\label{_vocabulary}}

According to Wikipedia,
\href{https://en.wikipedia.org/wiki/Forward_compatibility}{\textbf{forward
compatibility}} refers to the condition in which an old software can
process data created by newer softwares, ignoring the parts it doesn't
understand.

\begin{quote}
A standard supports forward compatibility if a product that complies
with earlier versions can "gracefully" process input designed for later
versions of the standard, ignoring new parts which it does not
understand.

---  Forward compatibility Wikipedia
\end{quote}

Vice versa,
\href{https://en.wikipedia.org/wiki/Backward_compatibility}{\textbf{backward
compatibility}} refers to when data from an old software is usable on
newer softwares. A change is said to be fully compatible if it's both
forward and backward compatible.

A change to the Bitcoin consensus rules is said to be a \textbf{soft
fork} if it is fully compatible. This is the most common way to upgrade
Bitcoin, for a number of reasons that we'll discuss further in this
chapter. If a change to the Bitcoin consensus rules is backward
compatible but not forward compatible, it is called a \textbf{hard
fork}.

For a technical overview of soft forks and hard forks, please read
\href{https://rosenbaum.se/book/grokking-bitcoin-11.html}{chapter 11 of
Grokking Bitcoin}. It explains these terms and also dives into the
upgrade mechanisms. It's recommended, although not strictly necessary,
to get a grip on this before you continue reading.

\hypertarget{historic-upgrades}{%
\subsection{Historic upgrades}\label{historic-upgrades}}

Bitcoin is not the same today as it was when the genesis block was
created. Several upgrades have been made throughout the years. In 2017,
Eric Lombrozo
\href{https://btctranscripts.com/breaking-bitcoin/2017/changing-consensus-rules-without-breaking-bitcoin/}{spoke
at the Breaking Bitcoin conference}
(\href{https://www.youtube.com/watch?v=0WCaoGiAOHE\&t=1926s}{video})
about Bitcoin's different upgrading mechanisms, pointing out how much
they have evolved over time. He even explained how Satoshi Nakamoto once
upgraded Bitcoin through a hard fork.

\begin{quote}
There was actually a hard-fork in bitcoin that Satoshi did that we would
never do it this way- it's a pretty bad way to do it. If you look at the
git commit description here
{[}\href{https://github.com/bitcoin/bitcoin/commit/757f0769d8360ea043f469f3a35f6ec204740446}{757f076}{]},
he says something about reverted makefile.unix wx-config version 0.3.6.
Right. That's all it says. It has no indication that it has a breaking
change at all. He was basically hiding it in there. He also
\href{https://bitcointalk.org/index.php?topic=626.msg6451\#msg6451}{posted
to bitcointalk} and said, please upgrade to 0.3.6 ASAP. We fixed an
implementation bug where it is possible that bogus transactions can be
displayed as accepted. Do not accept bitcoin payments until you upgrade
to 0.3.6. If you can't upgrade right away, then it would be best to
shutdown your bitcoin node until you do. And then on top of that, I
don't know why he decided to do this as well, he decided to add some
optimizations in the same code. Fix a bug and add some optimizations.

---  Eric Lombrozo Changing Consensus Rules Without Breaking Bitcoin at
Breaking Bitcoin conference (2017)
\end{quote}

He points out that, be it intentionally or not, this hard fork created
opportunities for future soft forks, namely the Script operators
(opcodes) OP\_NOP1-OP\_NOP10. We'll look more into
\protect\hyperlink{cve-2010-5141}{this code change} in chapter 9. These
opcodes have been used for two soft forks so far:
\href{https://github.com/bitcoin/bips/blob/master/bip-0065.mediawiki}{BIP65}
(OP\_CHECKLOCKTIMEVERIFY), and
\href{https://github.com/bitcoin/bips/blob/master/bip-0112.mediawiki}{BIP113}
(OP\_SEQUENCEVERIFY).

Lombrozo also provides an overview of the way upgrade mechanisms have
evolved throughout the years, up until 2017. Since then, only one other
major upgrade, which we analyze in
\protect\hyperlink{taproot-deployment}{Taproot upgrade - Speedy Trial},
has been deployed. The long and somewhat chaotic process that led to its
activation has helped us gain further insights on upgrading mechanisms
in Bitcoin.

\hypertarget{segwit-upgrade}{%
\subsubsection{Segwit upgrade}\label{segwit-upgrade}}

While all the upgrades preceding Segwit had been more or less painless,
this one was different. When Segwit activation code was released, in
October 2016, there seemed to be overwhelming support for it among
Bitcoin users, but for some reason miners didn't signal support for this
upgrade, which stalled the activation with no resolution in sight.

Aaron van Wirdum describes this winding road in his Bitcoin Magazine
article
\href{https://bitcoinmagazine.com/technical/the-long-road-to-segwit-how-bitcoins-biggest-protocol-upgrade-became-reality}{The
Long Road To Segwit}. He starts by explaining what Segwit is and how
that taps into the block size debate. Van Wirdum then outlines the turn
of events that led to its final activation. At the center of this
process was an upgrade mechanism called \emph{user activated soft fork},
or UASF for short, that was proposed by user Shaolinfry.

\begin{quote}
Shaolinfry proposed an alternative: a user activated soft fork (UASF).
Instead of hash power activation, a user activated soft fork would have
a ``\,`flag day activation' where nodes begin enforcement at a
predetermined time in the future.'' As long as such a UASF is enforced
by an economic majority, this should compel a majority of miners to
follow (or activate) the soft fork.

---  Aaron van Wirdum The Long Road To Segwit on Bitcoin Magazine (2017)
\end{quote}

Among other things, he cites Shaolinfry's email to the Bitcoin-dev
mailing list. In that occasion Shaolinfry
\href{https://lists.linuxfoundation.org/pipermail/bitcoin-dev/2017-February/013643.html}{argued
against miner activated soft forks}, listing a number of problems with
them.

\begin{quote}
Firstly, it requires trusting the hash power will validate after
activation. The BIP66 soft fork was a case where 95\% of the hashrate
was signaling readiness but in reality about half was not actually
validating the upgraded rules and mined upon an invalid block by
mistake{[}1{]}.

Secondly, miner signalling has a natural veto which allows a small
percentage of hashrate to veto node activation of the upgrade for
everyone. To date, soft forks have taken advantage of the relatively
centralised mining landscape where there are relatively few mining pools
building valid blocks; as we move towards more hashrate
decentralization, it's likely that we will suffer more and more from
"upgrade inertia" which will veto most upgrades.

---  Shaolinfry Bitcoin-dev mailing list (2017)
\end{quote}

Shaolinfry also drew attention to a common misinterpretation of miner
signaling: people generally thought that it was a means by which miners
could decide upon protocol upgrades, rather than an action that helped
coordinate upgrades. Due to this misunderstanding, miners might have
also felt obliged to proclaim in public their views on a certain soft
fork, as if that gave weight to the proposal.

The UASF proposal is, in a nutshell, a ``flag day'' on which nodes start
enforcing specific new rules. That way, miners don't have to make a
collective effort to coordinate the upgrade, but \emph{can} trigger
activation earlier than the flag day if enough blocks signal support.

\begin{quote}
My suggestion is to have the best of both worlds. Since a user activated
soft fork needs a relatively long lead time before activation, we can
combine with BIP9 to give the option of a faster hash power coordinated
activation or activation by flag day, whichever is the sooner. In both
cases, we can leverage the warning systems in BIP9. The change is
relatively simple, adding an activation-time parameter which will
transition the BIP9 state to LOCKED\_IN before the end of the BIP9
deployment timeout.

---  Shaolinfry Bitcoin-dev mailing list (2017)
\end{quote}

This idea caught a lot of interest, but didn't seem to reach near
unanimous support, which caused concern for a potential chain split. The
article by Aaron van Wirdum explains how this finally got resolved
thanks to
\href{https://github.com/bitcoin/bips/blob/master/bip-0091.mediawiki}{BIP91},
authored by James Hilliard.

\begin{quote}
Hilliard proposed a slightly complex but clever solution that would make
everything compatible: Segregated Witness activation as proposed by the
Bitcoin Core development team, the BIP148 UASF and the New York
Agreement activation mechanism. His BIP91 could keep Bitcoin whole ---
at least throughout SegWit activation.

---  Aaron van Wirdum The Long Road To Segwit on Bitcoin Magazine (2017)
\end{quote}

There were some more complicating factors involved (e.g. the so-called
"New York Agreement"), that this BIP had to take into consideration. We
encourage you to read Van Wirdum's article in full to learn about the
many interesting details in this story.

\hypertarget{_post_segwit_discussion}{%
\subsubsection{Post-Segwit discussion}\label{_post_segwit_discussion}}

After the Segwit deployment, a discussion about deployment mechanisms
emerged. As noted by Eric Lombrozo in
\href{https://btctranscripts.com/breaking-bitcoin/2017/changing-consensus-rules-without-breaking-bitcoin/}{his
talk at the Breaking Bitcoin conference}
(\href{https://www.youtube.com/watch?v=0WCaoGiAOHE\&t=1926s}{video}) and
by Shaolinfry (see \protect\hyperlink{segwit-upgrade}{Segwit upgrade}
above), a miner activated soft fork isn't the ideal upgrade mechanism.

\begin{quote}
At some point we're probably going to want to add more features to the
bitcoin protocol. This is a big philosophical question we're asking
ourselves. Do we do a UASF for the next one? What about a hybrid
approach? Miner activated by itself has been ruled out. bip9 we're not
going to use again.

---  Eric Lombrozo Changing Consensus Rules Without Breaking Bitcoin at
Breaking Bitcoin conference (2017)
\end{quote}

In January 2020, Matt Corallo
\href{https://lists.linuxfoundation.org/pipermail/bitcoin-dev/2020-January/017547.html}{sent
an email} to the Bitcoin-dev mailing list that started a discussion on
future soft fork deployment mechanisms. He listed five goals that he
thought were essential in an upgrade. David Harding
\href{https://bitcoinops.org/en/newsletters/2020/01/15/\#discussion-of-soft-fork-activation-mechanisms}{summarizes
them in a Bitcoin Optech newsletter} as:

\begin{quote}
\begin{enumerate}
\def\labelenumi{\arabic{enumi}.}
\item
  The ability to abort if a serious objection to the proposed consensus
  rules changes is encountered
\item
  The allocation of enough time after the release of updated software to
  ensure that most economic nodes are upgraded to enforce those rules
\item
  The expectation that the network hash rate will be roughly the same
  before and after the change, as well as during any transition
\item
  The prevention, as much as possible, of the creation of blocks that
  are invalid under the new rules, which could lead to false
  confirmations in non-upgraded nodes and SPV clients
\item
  The assurance that the abort mechanisms can't be misused by griefers
  or partisans to withhold a widely desired upgrade with no known
  problems
\end{enumerate}

---  David Harding Bitcoin Optech newsletter \#80 (2020)
\end{quote}

What Corallo proposes is a combination of a miner activated soft fork
and a user activated soft fork:

\begin{quote}
Thus, as something a bit more concrete, I think an activation method
which sets the right precedent and appropriately considers the above
goals, would be:

1) a standard BIP 9 deployment with a one-year time horizon for
activation with 95\% miner readiness,\\
2) in the case that no activation occurs within a year, a six month
quieting period during which the community can analyze and discussion
the reasons for no activation and,\\
3) in the case that it makes sense, a simple command-line/bitcoin.conf
parameter which was supported since the original deployment release
would enable users to opt into a BIP 8 deployment with a 24-month
time-horizon for flag-day activation (as well as a new Bitcoin Core
release enabling the flag universally).

This provides a very long time horizon for more standard activation,
while still ensuring the goals in \#5 are met, even if, in those cases,
the time horizon needs to be significantly extended to meet the goals of
\#3. Developing Bitcoin is not a race. If we have to, waiting 42 months
ensures we're not setting a negative precedent that we'll come to regret
as Bitcoin continues to grow.

---  Matt Corallo Modern Soft Fork Activation on Bitcoin-dev mailing
list (2020)
\end{quote}

\hypertarget{taproot-deployment}{%
\subsubsection{Taproot upgrade - Speedy
Trial}\label{taproot-deployment}}

When Taproot was ready for deployment in October 2020, meaning all the
technical details around its consensus rules had been implemented and
had reached broad approval within the community, discussions on how to
actually deploy it started to heat up. These discussions had been pretty
low key up until that point.

Lots of proposals for activation mechanisms started floating around, and
David Harding
\href{https://en.bitcoin.it/wiki/Taproot_activation_proposals}{summarized
them on the Bitcoin Wiki}. In his article he explained some properties
of BIP8, which at that time had some recent changes made in order to
make it more flexible.

\begin{quote}
At the time this document is being written,
\href{https://github.com/bitcoin/bips/blob/master/bip-0008.mediawiki}{BIP8}
has been drafted based on lessons learned in 2017. One notable change
following BIPs 9+148 is that forced activation is now based on block
height rather than median time past; a second notable change is that
forced activation is a boolean parameter chosen when a soft fork's
activation parameters are set either for the initial deployment or
updated in a later deployment.

BIP8 without forced activation is very similar to
\href{https://github.com/bitcoin/bips/blob/master/bip-0009.mediawiki}{BIP9}
version bits with timeout and delay, with the only significant
difference being BIP8's use of block heights compared to BIP9's use of
median time past. This setting allows the attempt to fail (but it can be
retried later).

BIP8 with forced activation concludes with a mandatory signaling period
where all blocks produced in compliance with its rules must signal
readiness for the soft fork in a way that will trigger activation in an
earlier deployment of the same soft fork with non-mandatory activation.
In other words, if node version x is released without forced activation
and, later, version y is released that successfully forces miners to
begin signaling readiness within the same time period, both versions
will begin enforcing the new consensus rules at the same time.

This flexibility of the revised BIP8 proposal makes it possible to
express some other ideas in terms of what they would look like using
BIP8. This provides a common factor to use for categorizing many
different proposals.

---  David Harding Taproot Activation Proposals on the Bitcoin Wiki
(2020)
\end{quote}

From this point forward the discussions became very heated, especially
around whether \texttt{lockinontimeout} should be \texttt{true} (as in a
user activated soft fork, referred to as ``BIP8 with forced activation''
by Harding) or \texttt{false} (as in a miner activated soft fork,
referred to as ``BIP8 without forced activation'' by Harding).

Among the proposals listed, one of them was titled ``Let's see what
happens''. For some reason, this proposal didn't get much traction until
seven months later.

During those seven months, the discussion went on and it seemed like
there was no way to reach broad consensus over which deployment
mechanism to use. There were mainly two camps: one that preferred
\texttt{lockinontimeout=true} (the UASF crowd) and the other one that
preferred \texttt{lockinontimeout=false} (the ``try and if it fails
rethink'' crowd). Since there was no overwhelming support for any of
these options, the debate went in circles with seemingly no way forward.
Some of these discussions were held on IRC, in a channel called
\#\#taproot-activation, but
\href{https://gnusha.org/taproot-activation/2021-03-05.log}{on March 5th
2021}, something changed:

\begin{quote}
\begin{verbatim}
06:42 < harding> roconnor: is somebody proposing BIP8(3m, false)?  I mentioned that the other day but I didn't see any responses.
 [...]
06:43 < willcl_ark_> Amusingly, I was just thinking to myself that, vs this, the SegWit activation was actually pretty straightforward: simply a LOT=false and if it fails a UASF.
06:43 < maybehuman> it's funny, "let's see what happens" (i.e. false, 3m) was a poular choice right at the beginning of this channel iirc
06:44 < roconnor> harding: I think I am.  I don't know how much that is worth.  Mostly I think it would be a widely acceptable configuration based on my understanding of everyone's concerns.
06:44 < willcl_ark_> maybehuman: becuase everybody actually wants this, even miners reckoned they could upgrade in about two weeks (or at least f2pool said that)
06:44 < roconnor> harding: BIP8(3m,false) with an extended lockin-period.
06:45 < harding> roconnor: oh, good.  It's been my favorite option since I first summarized the options on the wiki like seven months ago.
06:45 <@michaelfolkson> UASF wouldn't release (true,3m) but yeah Core could release (false, 3m)
06:45 < willcl_ark_> harding: It certainly seems like a good approach to me. _if_ that fails, then you can try an understand why, without wasting too much time
\end{verbatim}

---  \#taproot-activation IRC log
\end{quote}

The ``let's see what happens'' approach finally seemed to click in
peoples' minds. This process would later be labeled as ``Speedy Trial''
due to its short signaling period. David Harding explains this idea to
the broader community in an
\href{https://lists.linuxfoundation.org/pipermail/bitcoin-dev/2021-March/018583.html}{email
to the Bitcoin-dev mailing list}.

\begin{quote}
The earlier version of this proposal was documented over 200 days
ago{[}3{]} and taproot's underlying code was merged into Bitcoin Core
over 140 days ago.{[}4{]} If we had started Speedy Trial at the time
taproot was merged (which is a bit unrealistic), we would've either be
less than two months away from having taproot or we would have moved on
to the next activation attempt over a month ago.

Instead, we've debated at length and don't appear to be any closer to
what I think is a widely acceptable solution than when the mailing list
began discussing post-segwit activation schemes over a year ago.{[}5{]}
I think Speedy Trial is a way to generate fast progress that will either
end the debate (for now, if activation is successful) or give us some
actual data upon which to base future taproot activation proposals.

---  David Harding on Bitcoin-dev mailing list
\end{quote}

This deployment mechanism was refined over the course of two months and
then released in
\href{https://github.com/bitcoin/bitcoin/blob/master/doc/release-notes/release-notes-0.21.1.md\#taproot-soft-fork}{Bitcoin
Core version 0.21.1}. The miners quickly started signaling for this
upgrade moving the deployment state to \texttt{LOCKED\_IN}, and after a
grace period the Taproot rules were activated mid-November 2021 in block
\href{https://mempool.space/block/0000000000000000000687bca986194dc2c1f949318629b44bb54ec0a94d8244}{709632}.

\hypertarget{_future_deployment_mechanisms}{%
\subsubsection{Future deployment
mechanisms}\label{_future_deployment_mechanisms}}

Given the problems with the recent soft forks, Segwit and Taproot, it's
not clear how the next upgrade will be deployed. Speedy Trial was used
to deploy Taproot, but it was used to bridge the chasm between the UASF
and the MASF crowds, not because it has emerged as the best known
deployment mechanism.

\hypertarget{upgrading-risks}{%
\subsection{Risks}\label{upgrading-risks}}

During the activation of any fork, be it hard or soft, miner activated
or user activated, there's the risk of a long-lasting chain split. A
split that lingers for more than a few blocks can cause severe damage to
the sentiment around Bitcoin as well as to its price. But above all, it
would cause great confusion over what Bitcoin is. Is Bitcoin this chain
or that chain?

The risk with a user activated soft fork is that the new rules get
activated even if the majority of the hash power doesn't support them.
This scenario would result in a long-lasting chain split, which would
persist until the majority of the hash power adopts the new rules. It
could be especially hard to incentivize miners to switch to the new
chain if they had already mined blocks after the split on the old chain,
because by switching branch they would be abandoning their own block
rewards. However, it's worth mentioning a remarkable episode: in March
2013 a long-lasting split, explained in
\protect\hyperlink{march2013split}{2013-03-11 DB locks issue 0.7.2 -
0.8.0 (CVE-2013-3220)}, occurred due to an unintentional hard fork and,
contrary to this incentive, two major mining pools made the decision to
abandon their branch of the split in order to restore consensus.

On the other hand, the risk with a miner activated soft fork is a
consequence of the fact that miners can engage in false signaling, which
means that the actual share of the hash power that supports the change
could be smaller than it looks. If the actual support doesn't comprise a
majority of the hash power, we'd probably see a long-lasting chain split
similar to the one described in the previous paragraph. This, or at
least a similar issue, has happened in reality when BIP66 was deployed
(see \protect\hyperlink{bip66-splits}{BIP66}), but it got resolved
within 6 blocks or so.

\hypertarget{_costs_of_a_split}{%
\subsubsection{Costs of a split}\label{_costs_of_a_split}}

Jimmy Song
\href{https://btctranscripts.com/breaking-bitcoin/2017/socialized-costs-of-hard-forks/}{spoke
about the costs associated with hard forks} at Breaking Bitcoin in
Paris, but much of what he said applies to a chain split due to a failed
soft fork as well. He spoke about \emph{negative externalities}, and
defined them as the price someone else has to pay for your own actions.

\begin{quote}
The classic example of a negative externality is a factory. Maybe they
are producing-- maybe it's an oil refinery and they produce a good that
is good for the economy but they also produce something that is a
negative externality, like pollution. It's not just something that
everyone has to pay for, to clean up, or suffer from. But it's also 2nd
and 3rd order effects, like more traffic going towards the factory as a
result of more workers that need to go there. You might also have- you
might endanger some wildlife around there. It's not that everyone has to
pay for the negative externalities, it might be specific people, like
people who were previously using that road or animals that were near
that factory, and they are also paying for the cost of that factory.

---  Jimmy Song Socialized Costs Of Hard Forks at Breaking Bitcoin
conference (2017)
\end{quote}

In the context of Bitcoin, he exemplifies negative externalities using
Bitcoin Cash (bcash), which is a hard fork of Bitcoin created shortly
prior to that conference in 2017. He categorizes the negative
externalities of a hard fork into one-time costs and permanent costs.

Among the many examples of one-time costs, he mentions the ones incurred
by exchanges.

\begin{quote}
So we have a bunch of exchanges and they had a lot of one-time costs
that they had to pay. The first thing that happened is that deposits and
withdrawals had to be halted for a day or two for these exchanges
because they didn't know what would happen. Many of these exchanges had
to dip into cold storage because their users were demanding bcash. It's
part of their fidicuiary duty, they have to do that. You also have to
audit the new software. This is something that we had to do at itbit. We
want to spend bcash- how do we do it? We have to download electron cash?
Does it have malware? We have to go and audit it. We had like 10 days to
figure out if this was okay or not. And then you have to decide, are we
going to just allow a one-time withdrawal, or are we going to list this
new coin? For an exchange to lis ta new coin, it's not easy- there's all
sorts of new procedures for cold storage, signing, deposits,
withdrawals. Or you could just have this one-off event where you give
them their bcash at some point and then you never think about it again.
But that has its problems too. And finally, and whatever way you do it,
withdrawals or listing-- you are going to need new infrastructure to
work with this token in some way, even if it's a one-time withdrawal.
You need some way to give these tokens to your users. Again,
short-notice. Right? No time to do this, has to be done quickly.

---  Jimmy Song Socialized Costs Of Hard Forks at Breaking Bitcoin
conference (2017)
\end{quote}

He also lists the one-time costs incurred by merchants, payment
processors, wallets, miners, and users, as well as some of the permanent
costs, for example privacy loss and a higher risk of reorgs.

Indeed, when a split happens and the chain with the most general rules
becomes stronger than the chain with the stricter rules, a reorg will
occur. This will have a severe impact on all transactions carried out in
the wiped-out branch. For these reasons it's really important to try
avoiding chain splits at all times.

\hypertarget{_conclusion_3}{%
\subsection{Conclusion}\label{_conclusion_3}}

Bitcoin grows and evolves with time. Different upgrade mechanisms have
been used over the years and the learning curve is steep. More and more
sophisticated and robust methods keep being invented, as we learn more
about how the network reacts.

To keep Bitcoin in harmony, soft forks have proven to be the way
forward, but the big question is still not fully answered: how do we
safely deploy soft forks without causing discord?

\hypertarget{adversarialthinking}{%
\section{Adversarial thinking}\label{adversarialthinking}}

\includegraphics{images/adversarialthinking-banner.jpg}

This chapter addresses \emph{adversarial thinking}, a mindset that
focuses on what could go wrong and how adversaries might act. We start
out by discussing Bitcoin's security assumptions and security model,
after which we explain how ordinary users can improve their
self-sovereignty and Bitcoin's full node decentralization by thinking
adversarially. Then, we look into some actual threats to Bitcoin as well
as into the adversary's mind. Lastly, we talk about the \emph{axiom of
resistance} which can help you understand why people are working on
Bitcoin in the first place.

When discussing security within various systems, it's important to
understand what the security assumptions are. A typical security
assumption in Bitcoin is ``the discrete logarithm problem is hard to
solve'', which, simply put, means it's practically impossible to find a
private key that corresponds to a particular public key. Another pretty
strong security assumption is that a majority of the network's hashpower
is honest, meaning that they play by the rules. If these assumptions are
proven wrong, then Bitcoin is in trouble.

In 2015 Andrew Poelstra
\href{https://btctranscripts.com/scalingbitcoin/hong-kong-2015/security-assumptions/}{gave
a talk} at the Scaling Bitcoin conference in Hong Kong, during which he
analyzed Bitcoin's security assumptions. He starts by noticing that many
systems disregard adversaries to some extent; for example, it's really
hard to protect a building against all types of adversarial events.
Instead, we generally accept the possibility that someone may burn the
building down, and to some extent prevent this and other adversarial
behaviors through law enforcement etc.

But online things are different:

\begin{quote}
However, online we don't have this. We have pseudonymous and anonymous
behavior, anyone can connect to everyone and hurt the system. If it's
possible to adversarially hurt the system, then they will do it. We
cannot assume they will be visible and that they will be caught.

---  Andrew Poelstra Security Assumptions at Scaling Bitcoin Hong Kong
(2015)
\end{quote}

The consequence is that all known weaknesses in Bitcoin must somehow be
taken care of, otherwise they will be exploited. After all, Bitcoin is
the greatest honey pot in the world.

Poelstra goes on to mention how Bitcoin is a new kind of system; it's
more nebulous than, for example, a signing protocol which has very
clear-cut security assumptions.

On his personal blog, software engineer Jameson Lopp,
\href{https://blog.lopp.net/bitcoins-security-model-a-deep-dive/}{dives
into this}:

\begin{quote}
In reality, the bitcoin protocol was and is being built without a
formally defined specification or security model. The best that we can
do is to study the incentives and behavior of actors within the system
in order to better understand and attempt to describe it.

---  Jameson Lopp Bitcoin's Security Model: A Deep Dive (2016)
\end{quote}

So, we have a system that seems to be working in practice, but that we
can't formally prove to be secure. A proof is probably not possible due
to the complexity of the system itself.

\hypertarget{_not_only_for_bitcoin_experts}{%
\subsection{Not only for Bitcoin
experts}\label{_not_only_for_bitcoin_experts}}

The importance of adversarial thinking also extends to everyday Bitcoin
users to some degree, not only to hardcore Bitcoin developers and
experts. Ragnar Lifthasir mentions in a
\href{https://bitcoinwords.github.io/tweetstorm-on-adversarial-thinking}{tweetstorm}
how simplistic narratives around Bitcoin - for example, ``just HODL'' -
can be degrading to Bitcoin itself, and concludes by saying

\begin{quote}
To make Bitcoin and ourselves stronger we need to think like the
software engineers who contribute to Bitcoin. They peer review,
mercilessly seeking flaws. At their tech events they talk about every
which way a proposal can fail. They think adversarially. They're
conservative

---  Ragnar Lifthasir Twitter (2020)
\end{quote}

He refers to these simplistic narratives as monomanias. Through this
definition he's saying that by focusing on a single thing - for example,
``just HODL''- you risk to overlook the arguably more important stuff,
such as keeping your Bitcoin secure or doing your best to use Bitcoin in
a trustless manner.

\hypertarget{threats}{%
\subsection{Threats}\label{threats}}

There are a lot of known weaknesses in Bitcoin, and many of them are
actively being exploited. To get a glimpse of that, have a look at the
\href{https://en.bitcoin.it/wiki/Weaknesses}{Weaknesses page} on Bitcoin
wiki. There are mentioned a wide variety of problems, such as wallet
theft and denial-of-service attacks.

\begin{quote}
If an attacker attempts to fill the network with clients that they
control, you would then be very likely to connect only to attacker
nodes. Although Bitcoin never uses a count of nodes for anything,
completely isolating a node from the honest network can be helpful in
the execution of other attacks.

---  Various authors Bitcoin wiki
\end{quote}

This type of attack is called \emph{Sybil attack}, and it occurs
whenever a single entity controls multiple nodes in a network and uses
them to appear as multiple entities.

As the quote also mentions, the Sybil attack is not effective on the
Bitcoin network because there is no voting through nodes or other
numerable entities, but rather through computing power (see
\protect\hyperlink{minerdecentralization}{Miner decentralization}).
Nonetheless, this flat structure leaves the system susceptible to other
attacks. The Bitcoin wiki page also outlines other possible attacks,
such as information hiding (often referred to as \emph{eclipse attack}),
and the way Bitcoin Core implements some heuristic countermeasures
against such attacks.

The above are examples of real threats that need to be taken care of.

\begin{figure}
\hypertarget{fig-sabotage-manual}{%
\centering
\includegraphics{images/sabotage-manual.png}
\caption{Excerpt from the Simple Sabotage Field
Manual}\label{fig-sabotage-manual}
}
\end{figure}

To better understand the adversary's mind, it might be helpful to get a
glimpse into how they operate. A US government body named Office of
Strategic Services, which operated during World War II and had among its
purposes to conduct espionage, perform sabotage and spread propaganda,
produced a \href{https://www.gutenberg.org/ebooks/26184}{manual} for
their personnel on how to properly sabotage the enemy. Its title was
``Simple Sabotage Field Manual'' and contained concrete tips on
infiltrating the enemy to make their lives hard. The tips range from
burning down warehouses to causing wear to drills in order to decrease
the enemy's efficiency.

For example, there is a section
(\protect\hyperlink{fig-sabotage-manual}{figure\_title}) about how an
infiltrator can disrupt organizations. It's not hard to see how such
tactics could be used to target the Bitcoin development process (see
\protect\hyperlink{opensource}{Open Source}), which is open for anyone
to participate in. A dedicated attacker can keep stalling progress by
endless concerns of irrelevant issues, haggle over precise wordings, and
attempt to reiterate discussions that have already been comprehensively
addressed. The attacker can also hire a troll army to multiply their own
effectiveness; we can call this a social Sybil attack. Using a social
Sybil attack, they can make it look like there's more resistance against
a proposed change than there actually is.

This highlights how a determined state can and will do everything in its
power to destroy the enemy, including breaking it down from the inside.
Since Bitcoin is a form of money that competes with established fiat
currencies, chances are that states will regard Bitcoin as an enemy.

Eric Voskuil
\href{https://github.com/libbitcoin/libbitcoin-system/wiki/Axiom-of-Resistance}{writes
on his Cryptoeconomics wiki page} about what he calls the ``axiom of
resistance'':

\begin{quote}
In other words there is an assumption that it is \emph{possible} for a
system to resist state control. This is not accepted as a fact but
deemed to be a reasonable assumption, due to empirical study of behavior
of similar systems, on which to base the system.

\textbf{One who does not accept the axiom of resistance is contemplating
an entirely different system than Bitcoin.} If one assumes it is
\emph{not possible} for a system to resist state controls, conclusions
do not make sense in the context of Bitcoin - just as conclusions in
spherical geometry contradict Euclidean. How can Bitcoin be
permissionless or censorship-resistant without the axiom? The
contradiction leads one to make obvious errors in an attempt to
rationalize the conflict.

---  Eric Voskuil Cryptoeconomics wiki (2017)
\end{quote}

What he's essentially saying is that only when one assumes it's possible
to create a system that states can't control, is it meaningful to try.

This means that to work on Bitcoin you should accept the axiom of
resistance, otherwise you'd better spend your time on other projects.
Acknowledging that axiom helps you focusing your development efforts on
the real problems at hand: coding around state-level adversaries. In
other words, think adversarially.

\hypertarget{opensource}{%
\section{Open Source}\label{opensource}}

\includegraphics{images/opensource-banner.jpg}

Bitcoin is built using open source software. In this chapter we analyze
what this means, how maintenance of the software works, and how open
source software in Bitcoin allows for permissionless development. We dip
our toes into \emph{selection cryptography}, which deals with the
selection and use of libraries in cryptographic systems. The chapter
includes a section about Bitcoin's review process, followed by another
one on the ways Bitcoin developers get funded. The last section talks
about how Bitcoin's open source culture can look really weird from the
outside, and why this perceived weirdness is really a sign of good
health.

Most Bitcoin softwares, and especially Bitcoin Core, is open source.
This means that the source code of the software is made available to the
general public for scrutiny, tinkering, modification, and
redistribution. The definition of open source at
\url{https://opensource.org/osd} includes, among others, the following
important points:

\begin{quote}
\begin{description}
\item[Free Redistribution]
The license shall not restrict any party from selling or giving away the
software as a component of an aggregate software distribution containing
programs from several different sources. The license shall not require a
royalty or other fee for such sale.
\item[Source Code]
The program must include source code, and must allow distribution in
source code as well as compiled form. Where some form of a product is
not distributed with source code, there must be a well-publicized means
of obtaining the source code for no more than a reasonable reproduction
cost, preferably downloading via the Internet without charge. The source
code must be the preferred form in which a programmer would modify the
program. Deliberately obfuscated source code is not allowed.
Intermediate forms such as the output of a preprocessor or translator
are not allowed.
\item[Derived Works]
The license must allow modifications and derived works, and must allow
them to be distributed under the same terms as the license of the
original software.
\end{description}

---  The Open Source Definition Open Source Initiative website
\end{quote}

Bitcoin Core adheres to this definition by being distributed under the
\href{https://github.com/bitcoin/bitcoin/blob/master/COPYING}{MIT
License}:

\begin{verbatim}
The MIT License (MIT)

Copyright (c) 2009-2022 The Bitcoin Core developers
Copyright (c) 2009-2022 Bitcoin Developers

Permission is hereby granted, free of charge, to any person obtaining a copy
of this software and associated documentation files (the "Software"), to deal
in the Software without restriction, including without limitation the rights
to use, copy, modify, merge, publish, distribute, sublicense, and/or sell
copies of the Software, and to permit persons to whom the Software is
furnished to do so, subject to the following conditions:

The above copyright notice and this permission notice shall be included in
all copies or substantial portions of the Software.
\end{verbatim}

As noted in \protect\hyperlink{donttrustverify}{Don't trust, verify},
it's important for users to be able to verify that the Bitcoin software
they run ``works as advertised''. To do that, they must have
unrestricted access to the source code of the software they wish to
verify.

In the upcoming sections we dive into some other interesting aspects of
open source software in Bitcoin.

\hypertarget{softwaremaintenance}{%
\subsection{Software maintenance}\label{softwaremaintenance}}

Bitcoin Core's source code is maintained in a Git repository hosted on
\href{https://github.com/bitcoin/bitcoin}{GitHub}. Anyone can clone that
very repository without asking for any permission, and then inspect,
build, or make changes to it locally. This means that there are many
thousands of copies of the repository spread throughout the globe. These
are all copies of the same repository, so what makes this specific
GitHub Bitcoin Core repository so special? Technically it's not special
at all, but socially it has become the focal point of Bitcoin
development.

Bitcoin and security expert Jameson Lopp explains this very well in a
\href{https://blog.lopp.net/who-controls-bitcoin-core-/}{blog post}
titled ``Who Controls Bitcoin Core?'':

\begin{quote}
Bitcoin Core is a focal point for development of the Bitcoin protocol
rather than a point of command and control. If it ceased to exist for
any reason, a new focal point would emerge --- the technical
communications platform upon which it's based (currently the GitHub
repository) is a matter of convenience rather than one of definition /
project integrity. In fact, we have already seen Bitcoin's focal point
for development change platforms and even names!

---  Jameson Lopp Who Controls Bitcoin Core? (2018)
\end{quote}

He goes on to explain how Bitcoin Core's software is maintained and
secured against malicious code changes. The general takeaway from this
full article is summarized at its very end:

\begin{quote}
No one controls Bitcoin.

No one controls the focal point for Bitcoin development.

---  Jameson Lopp Who Controls Bitcoin Core? (2018)
\end{quote}

Bitcoin Core developer Eric Lombrozo talks further about the development
process in his
\href{https://medium.com/@elombrozo/the-bitcoin-core-merge-process-74687a09d81d}{Medium
post} titled ``The Bitcoin Core Merge Process''.

\begin{quote}
Anyone can fork the code base repository and make arbitrary changes to
their own repository. They can build a client from their own repository
and run that instead if they want. They can also make binary builds for
other people to run.

If someone wants to merge a change they've made in their own repository
into Bitcoin Core, they can submit a pull request. Once submitted,
anyone can review the changes and comment on them regardless of whether
or not they have commit access to Bitcoin Core itself.

---  Eric Lombrozo on Medium.com The Bitcoin Core Merge Process (2017)
\end{quote}

It should be noted that pull requests can take a very long time before
being merged to the repository by maintainers, and that's usually due to
a lack of review, see \protect\hyperlink{review}{Review}, which is often
due to a lack of \emph{reviewers}.

Lombrozo also talks about the process that surrounds consensus changes,
but that's a bit beyond the scope of this chapter. See
\protect\hyperlink{upgrading}{Upgrading} for more information on how the
Bitcoin protocol gets upgraded.

\hypertarget{_permissionless_development}{%
\subsection{Permissionless
development}\label{_permissionless_development}}

We've established that anyone can write code for Bitcoin Core without
asking for any permission, but not necessarily have it merged to the
main Git repository. This affects any modification, from changing color
schemes of the graphical user interface, to the way peer-to-peer
messages are formatted, and even consensus rules, i.e. the set of rules
that define a valid blockchain.

Probably equally important is that users are free to develop systems on
top of Bitcoin, without asking for any permission. We've seen countless
successful software projects that were built on top of Bitcoin, such as:

\begin{description}
\item[Lightning Network]
A payment network that allows for fast payment of very small amounts. It
requires very few on-chain Bitcoin transactions. Various inter-operable
implementations exist, such as
\href{https://github.com/ElementsProject/lightning}{Core Lightning},
\href{https://github.com/lightningnetwork/lnd}{LND},
\href{https://github.com/ACINQ/eclair}{Eclair}, and
\href{https://github.com/lightningdevkit}{Lightning Dev Kit}.
\item[CoinJoin]
Multiple parties collaborate to combine their payments into a single
transaction to make address clustering (explained in
\protect\hyperlink{blockchainprivacy}{Blockchain privacy}) harder.
Various implementations exist.
\item[Sidechains]
This system can lock a coin on Bitcoin's blockchain in order to unlock
it on some other blockchain. This allows for bitcoins to be moved to
some other blockchain, namely a sidechain, so as to use the features
available on that sidechain. Examples include
\href{https://github.com/ElementsProject/elements}{Blockstream's
Elements}.
\item[OpenTimestamps]
It allows you to \href{https://opentimestamps.org/}{timestamp a
document} on Bitcoin's blockchain in a private way. You can then use
that timestamp to prove that a document must have existed prior to a
certain time.
\end{description}

Without permissionless development, many of these projects would not
have been possible. As stated in
\protect\hyperlink{neutrality}{Neutrality}, if developers had to ask for
permission to build protocols on top of Bitcoin, only the protocols
allowed by the central developer granting committee would be developed.

It is common for systems like the ones listed above to be themselves
licensed as open source software, which in turn allows for people to
contribute, re-use, or review their code without asking for any
permission. Open source has become the gold standard of Bitcoin software
licensing.

\hypertarget{_pseudonymous_development}{%
\subsection{Pseudonymous development}\label{_pseudonymous_development}}

Not having to ask for permission to develop Bitcoin software brings an
interesting and important option to the table: you can write and publish
code, in Bitcoin Core or any other open source project, without
revealing your identity.

Many developers choose this option by operating under a pseudonym and
trying to keep it detached from their true identity. The reasons for
doing this can vary from developer to developer. One pseudonymous user
is ZmnSCPxj. Among other projects, he contributes to Bitcoin Core and
Core Lightning, one of several implementations of Lightning Network.
\href{https://zmnscpxj.github.io/about.html}{He writes} on his web page:

\begin{quote}
I am ZmnSCPxj, a randomly-generated Internet person. My pronouns are
he/him/his.

I understand that humans instinctively desire to know my identity.
However, I think my identity is largely immaterial, and prefer to be
judged by my work.

If you are wondering whether to donate or not, and wondering what my
cost of living or my income is, please understand that properly
speaking, you should donate to me based on the utility you find my
articles and my work on Bitcoin and the Lightning Network.

---  ZmnSCPxj on his GitHub page
\end{quote}

In his case, the reason for using a pseudonym is to be judged on his
merits and not on who the person or persons behind the pseudonym is or
are. Interestingly, he revealed in an
\href{https://www.coindesk.com/markets/2020/06/29/many-bitcoin-developers-are-choosing-to-use-pseudonyms-for-good-reason/}{article
on CoinDesk} that the pseudonym was created for a different reason.

\begin{quote}
My initial reason {[}for using a pseudonym{]} was simply that I was
concerned {[}about{]} making a massive mistake; thus ZmnSCPxj was
originally intended to be a disposable pseudonym that could be abandoned
in such a case. However it seems to have garnered a mostly positive
reputation, so I have retained it

---  Many Bitcoin Developers Are Choosing to Use Pseudonyms -- For Good
Reason on CoinDesk (2021)
\end{quote}

Using a pseudonym indeed allows you to speak more freely without putting
your personal reputation at risk should you say something stupid or make
some big mistake. As it turned out, his pseudonym got very reputable and
in 2019
\href{https://twitter.com/spiralbtc/status/1204815615678177280}{he even
got a development grant}, which is in itself a testament to Bitcoin's
permissionless nature.

Arguably, the most well-known pseudonym in Bitcoin is Satoshi Nakamoto.
It's unclear why he chose to be pseudonymous, but with hindsight it was
probably a good decision for multiple reasons:

\begin{itemize}
\item
  As many people speculate that Nakamoto owns a lot of bitcoin, it's
  imperative for his financial and personal safety to keep his identity
  unknown.
\item
  Since his identity is unknown, there is no possibility of prosecuting
  anyone, which gives various government authorities a hard time.
\item
  There is no authoritative person to look up to, making Bitcoin more
  meritocratic and resilient against blackmailing.
\end{itemize}

Notice that these points don't just hold true for Satoshi Nakamoto, but
for anyone working in Bitcoin or holding significant amounts of the
currency, to varying degrees.

\hypertarget{selectioncryptography}{%
\subsection{Selection cryptography}\label{selectioncryptography}}

Open source developers often make use of open source libraries developed
by other people. This is a natural and awesome part of any healthy
ecosystem. But Bitcoin software deals with real money and, in light of
this, developers need to be extra careful when choosing which third
party libraries it should depend on.

In a philosophical
\href{https://btctranscripts.com/greg-maxwell/2015-04-29-gmaxwell-bitcoin-selection-cryptography/}{talk
about cryptography} (you may find the video
\href{https://youtu.be/Gs9lJTRZCDc?t=2236}{here}), Gregory Maxwell wants
to redefine the term ``cryptography'' which he believes to be too
narrow. He explains that fundamentally \emph{information wants to be
free}, and makes his definition of cryptography based on that:

\begin{quote}
\textbf{Cryptography} is the art and science we use to fight the
fundamental nature of information, to bend it to our political and moral
will, and to direct it to human ends against all chance and efforts to
oppose it.

---  Gregory Maxwell Bitcoin Selection Cryptography (2015)
\end{quote}

He then introduces the term \emph{selection cryptography}, referred to
as the art of selecting cryptographic tools, and explains why it is an
important part of cryptography. It revolves around how to select
cryptographic libraries, tools, and practices, or as he says ``the
cryptosystem of picking cryptosystems''.

Using concrete examples, he shows how selection cryptography can easily
go horribly wrong, and also proposes a list of questions you could ask
yourself when practicing it. Below is a distilled version of that list:

\begin{enumerate}
\def\labelenumi{\arabic{enumi}.}
\item
  Is the software intended for your purposes?
\item
  Are the cryptographic considerations being taken seriously?
\item
  The review process\ldots\hspace{0pt} is there one?
\item
  What is the experience of the authors?
\item
  Is the software documented?
\item
  Is the software portable?
\item
  Is the software tested?
\item
  Does the software adopt best practices?
\end{enumerate}

While this is not the ultimate guide to success, it can be very helpful
to go through these points when doing selection cryptography.

Due to the issues mentioned above by Maxwell, Bitcoin Core tries really
hard to
\href{https://github.com/bitcoin/bitcoin/blob/master/doc/dependencies.md}{minimize
its exposure to third party libraries}. Of course, you can't eradicate
all external dependencies, otherwise you'd have to write everything by
yourself, from font rendering to implementation of system calls.

\hypertarget{review}{%
\subsection{Review}\label{review}}

This section is named ``Review'', rather than ``Code review'', because
Bitcoin's security relies heavily on review at multiple levels, not just
source code. Moreover, different ideas require review at different
levels: a consensus rule change would require a deeper review at more
levels compared to a color scheme change or a typo fix.

On its way to final adoption, an idea usually flows through several
phases of discussion and review. Some of these phases are listed below:

\begin{enumerate}
\def\labelenumi{\arabic{enumi}.}
\item
  An idea is posted on the Bitcoin-dev mailing list
\item
  The idea is formalized into a Bitcoin Improvement Proposal (BIP)
\item
  The BIP is implemented in a pull request (PR) to Bitcoin Core
\item
  Deployment mechanisms are discussed
\item
  Some competing deployment mechanisms are implemented in pull requests
  to Bitcoin Core
\item
  Pull requests are merged to the master branch
\item
  Users choose whether to use the software or not
\end{enumerate}

At each of these phases people with different points of view and
backgrounds review the available information, be it the source code, a
BIP, or just a loosely described idea. The phases are usually not
performed in any strict top-down manner, indeed multiple phases can
happen simultaneously, and sometimes you go back and forth between them.
Different people may also provide feedback during different phases.

One of the most prolific code reviewers on Bitcoin Core is Jon Atack. He
wrote
\href{https://jonatack.github.io/articles/how-to-review-pull-requests-in-bitcoin-core}{a
blog post} about how to review pull requests in Bitcoin Core. He
emphasizes that a good code reviewer focuses on how to best add value.

\begin{quote}
As a newcomer, the goal is to try to add value, with friendliness and
humility, while learning as much as possible.

A good approach is to make it not about you, but rather "How can I best
serve?"

---  Jon Atack How to Review Pull Requests in Bitcoin Core (2020)
\end{quote}

He highlights the fact that review is a truly limiting factor in Bitcoin
Core. Lots of good ideas get stuck in a limbo where no review occurs,
pending. Notice that reviewing is not only beneficial to Bitcoin, but
also a great way to learn about the software while providing value to
it, at the same time. Atack's rule of thumb is to review 5-15 PRs before
making any PR of your own. Again, your focus should be on how to best
serve the community, not on how to get your own code merged. On top of
this, he stresses the importance of doing review at the right level: is
this the time for nits and typos, or does the developer need more of a
conceptually-oriented review?

\begin{quote}
A useful first question when beginning a review can be, "What is most
needed here at this time?" Answering this question requires experience
and accumulated context, but it is a useful question in deciding how you
can add the most value in the least time.

---  Jon Atack How to Review Pull Requests in Bitcoin Core (2020)
\end{quote}

The second half of the post consists of some useful hands-on technical
guidance on how to actually do the reviewing, and provides links to
important documentation for further reading.

Bitcoin Core developer and code reviewer Gloria Zhao has written
\href{https://github.com/glozow/bitcoin-notes/blob/master/review-checklist.md}{an
article} containing questions she usually asks herself during a review.
She also states what she considers to be a good review.

\begin{quote}
I personally think a good review is one where I've asked myself a lot of
pointed questions about the PR and been satisfied with the answers to
them.\\
\ldots\hspace{0pt}{[}snip{]}\ldots\hspace{0pt}\\
Naturally, I start with conceptual questions, then approach-related
questions, and then implementation questions. Generally, I personally
think it's useless to leave C++ syntax-related comments on a draft PR,
and would feel rude going back to "does this make sense" after the
author has addressed 20+ of my code organization suggestions.

---  Gloria Zhao Common PR Review Questions on GitHub (2022)
\end{quote}

Her idea that a good review should focus on what's most needed at a
specific point in time aligns well with Jon Atack's advice. She proposes
a list of questions that you may ask yourself at various levels of the
review process, but stresses that this list is not in any way exhaustive
nor a straight-out recipe. The list is illustrated with real-life
examples from GitHub.

\hypertarget{_funding}{%
\subsection{Funding}\label{_funding}}

Lots of people work with Bitcoin open source development, either for
Bitcoin Core or for other projects. Many do it in their spare time
without getting any compensation, but some developers are also getting
paid to do it.

Companies, individuals, and organizations who have an interest in
Bitcoin's continued success can donate funds to developers, either
directly or through organizations that in turn distribute the funds to
individual developers. The website polylunar.com has compiled a
\href{https://polylunar.com/bitcoin-grants-tracker/}{list of grants}
given out by a broad range of individuals, organizations, and companies.
There are also a number of Bitcoin-focused companies that hire skilled
developers to let them work full-time on Bitcoin.

\hypertarget{_culture_shock}{%
\subsection{Culture shock}\label{_culture_shock}}

People sometimes get the impression that there's a lot of infighting and
endless heated debates among Bitcoin developers, and that they are
incapable of making decisions.

For example, the Taproot deployment mechanism, described in
\protect\hyperlink{taproot-deployment}{Taproot upgrade - Speedy Trial},
was discussed over a long period of time during which two ``camps''
formed. One wanted to ``fail'' the upgrade if miners hadn't
overwhelmingly voted for the new rules after a certain moment, while the
other wanted to enforce the rules after that moment no matter what.
Michael Folkson summarizes the arguments from the two camps in an
\href{https://lists.linuxfoundation.org/pipermail/bitcoin-dev/2021-February/018380.html}{email}
to the Bitcoin-dev mailing list.

The debate went on seemingly forever, and it was really hard to see any
consensus on this forming any time soon. This got people frustrated and
as a result the heat intensified. Gregory Maxwell (as user nullc)
worried
\href{https://www.reddit.com/r/Bitcoin/comments/hrlpnc/technical_taproot_why_activate/fyqbn8s/?utm_source=share\&utm_medium=web2x\&context=3}{on
Reddit} that the lengthy discussions would make the upgrade less safe.

\begin{quote}
At this juncture, additional waiting isn't adding more review and
certainty. Instead, additional delay is sapping inertia and potentially
increasing risk somewhat as people start forgetting details, delaying
work on downstream usage (like wallet support), and not investing as
much additional review effort as they would be investing if they felt
confident about the activation timeframe.

---  Gregory Maxwell on Reddit Is Taproot development moving too fast or
too slow?
\end{quote}

Eventually, this dispute got resolved thanks to a new proposal by David
Harding and Russel O'Connor called Speedy Trial, which entailed a
comparatively shorter signaling period for miners to lock in activation
of Taproot, or fail fast. If they activated it during that window of
time, then Taproot would be deployed approximately 6 months later. This
upgrade is covered in more detail in
\protect\hyperlink{upgrading}{Upgrading}.

Someone who's not used to Bitcoin's development process would probably
think that these heated debates look awfully bad and even toxic. There
are at least two factors that make them look bad, in some people's eyes:

\begin{itemize}
\item
  Compared to closed source companies, all debates happen in the open,
  unedited. A software company like Google would never let its employees
  debate proposed features in the open, indeed it would at most publish
  a statement about the company's stance on the subject. This makes
  companies look more harmonic compared to Bitcoin.
\item
  Since Bitcoin is permissionless, anyone is allowed to voice their
  opinions. This is fundamentally different from a closed source company
  that has a handful of people with an opinion, usually like-minded
  people. The plethora of opinions expressed within Bitcoin is simply
  staggering compared to, for example, PayPal.
\end{itemize}

Most Bitcoin developers would argue that this openness brings about a
good and healthy environment, and even that it is necessary for
producing the best outcome.

As hinted in \protect\hyperlink{threats}{Threats}, the second bullet
above can be very beneficial but comes with a downside. An attacker
could use stalling tactics, like the ones outlined in the
\href{https://www.gutenberg.org/ebooks/26184}{Simple Sabotage Field
Manual}, to distort the decision making and development process.

Another thing worth mentioning is that, as noted in
\protect\hyperlink{selectioncryptography}{Selection cryptography}, since
Bitcoin is money and Bitcoin Core secures unfathomable amounts of money,
security in this context is not taken lightly. This is why seasoned
Bitcoin Core developers might appear very hard-headed, which attitude is
usually warranted. Indeed, a feature with a weak rationale behind it is
not going to be accepted. The same would happen if it broke the
reproducible builds (described in
\protect\hyperlink{donttrustverify}{Don't trust, verify}), added new
dependencies, or if the code didn't follow Bitcoin's
\href{https://github.com/bitcoin/bitcoin/blob/master/doc/developer-notes.md}{best
practices}.

New (and old) developers can get frustrated by this. But, as is
customary in open source software, you can always fork the repository,
merge whatever you want to your own fork, and build and run your own
binary.

\hypertarget{scaling}{%
\section{Scaling}\label{scaling}}

\includegraphics{images/scaling-banner.jpg}

In this chapter, we explore how Bitcoin does and does not scale. We
start by looking at how people have reasoned about scaling in the past.
Then, the bulk of this chapter explains various approaches to scaling
Bitcoin, specifically vertical, horizontal, inward, and layered scaling.
Each description is followed by considerations over whether the approach
interferes with Bitcoin's value proposition.

In the Bitcoin space, different people ascribe different definitions to
the word ``scale''. Some conceive it as the increase of the blockchain
transaction capacity, others believe it equals to using the blockchain
more efficiently, and others see it as the development of systems on top
of Bitcoin.

In the context of Bitcoin, and for this book's purposes, we define
scaling as \emph{increasing Bitcoin's usage capacity without
compromising its censorship resistance}. This definition encompasses
several kinds of changes, for example:

\begin{itemize}
\item
  Making transaction inputs use fewer bytes
\item
  Improving signature verification performance
\item
  Making the peer-to-peer network use less bandwidth
\item
  Transaction batching
\item
  Layered architecture
\end{itemize}

We'll soon dive into different approaches to scaling, but let's start
with a brief overview of Bitcoin's history within the context of
scaling.

\hypertarget{_history}{%
\subsection{History}\label{_history}}

Scaling has been a focal point of discussion since the genesis of
Bitcoin. The very first sentence of the
\href{https://www.metzdowd.com/pipermail/cryptography/2008-November/014814.html}{very
first email} in response to Satoshi's announcement of the Bitcoin
whitepaper on the Cryptography mailing list was indeed about scaling:

\begin{quote}
Satoshi Nakamoto wrote:\\
\textgreater{} I've been working on a new electronic cash system that's
fully\\
\textgreater{} peer-to-peer, with no trusted third party.\\
\textgreater{}\\
\textgreater{} The paper is available at:\\
\textgreater{} \url{http://www.bitcoin.org/bitcoin.pdf}

We very, very much need such a system, but the way I understand your
proposal, it does not seem to scale to the required size.

---  James A. Donald and Satoshi Nakamoto Cryptography mailing list
(2008)
\end{quote}

The conversation in itself might not be very interesting nor accurate,
but it shows that scaling has been a concern from the very beginning.

Discussions over scaling reached their peak interest around 2015-2017,
when there were many different ideas circulating about whether and how
to increase the maximum block size limit. That was a rather
uninteresting discussion about changing a parameter in the source code,
a change that didn't fundamentally solve anything but pushed the problem
of scaling further into the future, building technical debt.

In 2015, a conference called \href{https://scalingbitcoin.org/}{Scaling
Bitcoin} was held in Montreal, with a follow-up conference six months
later in Hong Kong and thereafter in a number of other locations around
the world. The focus was precisely on how to address scaling. Many
Bitcoin developers and other enthusiasts gathered at these conferences
to discuss various scaling issues and proposals. Most of these
discussions didn't revolve around block size increases but on more
long-term solutions.

After the Hong Kong conference in December 2015, Gregory Maxwell
\href{https://lists.linuxfoundation.org/pipermail/bitcoin-dev/2015-December/011865.html}{summarized
his view} on many of the issues that had been debated, starting off with
some general scaling philosophy.

\begin{quote}
With the available technology, there are fundamental trade-offs between
scale and decentralization. If the system is too costly people will be
forced to trust third parties rather than independently enforcing the
system's rules. If the Bitcoin blockchain's resource usage, relative to
the available technology, is too great, Bitcoin loses its competitive
advantages compared to legacy systems because validation will be too
costly (pricing out many users), forcing trust back into the system. If
capacity is too low and our methods of transacting too inefficient,
access to the chain for dispute resolution will be too costly, again
pushing trust back into the system.

---  Gregory Maxwell Capacity increases for the Bitcoin system (2015)
\end{quote}

He speaks about the trade-off between throughput and decentralization.
If you allow for bigger blocks, you will push some people off the
network because they won't have the resources to validate the blocks
anymore. But on the other hand, if access to block space becomes more
expensive, fewer people will be able to afford using it as a dispute
resolution mechanism. In both cases, users are pushed towards trusted
services.

He continues by summarizing the many approaches to scaling presented at
the conference. Among them are more computationally efficient signature
verifications, \emph{segregated witness} including a block size limit
change, a more space-efficient block propagation mechanism, and building
protocols on top of Bitcoin in layers. Many of these approaches have
since been implemented.

\hypertarget{_scaling_approaches}{%
\subsection{Scaling approaches}\label{_scaling_approaches}}

As hinted above, scaling Bitcoin doesn't necessarily have to be about
increasing the block size limit or other limits. We now go through some
general approaches to scaling, some of which don't suffer from the
throughput-decentralization trade-off mentioned in the previous section.

\hypertarget{verticalscaling}{%
\subsubsection{Vertical scaling}\label{verticalscaling}}

Vertical scaling is the process of increasing the computing resources of
the machines processing data. In the context of Bitcoin, these latter
would be the full nodes, namely the machines that validate the
blockchain on behalf of their users.

The most commonly discussed technique for vertical scaling in Bitcoin is
the increase in the block size limit. This would require some full nodes
to upgrade their hardware to keep up with the increasing computational
demands. The downside is that it happens at the cost of centralization,
as was discussed in the previous section and more in depth in
\protect\hyperlink{fullnodedecentralization}{Full node
decentralization}.

Besides the negative effects on full node decentralization, vertical
scaling might also negatively impact Bitcoin's mining decentralization
(explained in \protect\hyperlink{minerdecentralization}{Miner
decentralization}) and security in less obvious ways. Let's have a look
at how miners ``should'' operate. Say a miner mines a block at height 7
and publishes that block on the Bitcoin network. It will take some time
for this block to reach broad acceptance, which is mainly due to two
factors:

\begin{itemize}
\item
  Transfer of the block between peers takes time due to bandwidth
  limitations.
\item
  Validation of the block takes time.
\end{itemize}

While block 7 is being propagated through the network, many miners are
still mining on top of block 6 because they haven't received and
validated block 7 yet. During this time, if any of these miners finds a
new block at height 7, there will be two competing blocks at that
height. There can only be one block at height 7 (or any other height),
which means one of the two candidates must become stale.

In short, stale blocks happen because it takes time for each block to
propagate, and the longer propagation takes, the higher the probability
of stale blocks.

Suppose that the block size limit is lifted and that the average block
size increases substantially. Blocks would then propagate slower across
the network due to bandwidth limitations and verification time. An
increase in propagation time will also increase the chances of stale
blocks.

Miners don't like to have their blocks staled because they'll lose their
block reward, so they will do whatever they can to avoid this scenario.
The measures they can take include:

\begin{itemize}
\item
  Postponing the validation of an incoming block, also known as
  \emph{validationless mining}, further discussed in
  \protect\hyperlink{bip66splits}{Splits due to validationless mining}.
  Miners can just check the block header's proof-of-work and mine on top
  of it, while in the meantime they download the full block and validate
  it.
\item
  Connecting to a mining pool with greater bandwidth and connectivity.
\end{itemize}

Validationless mining further undermines full node decentralization, as
the miner resorts to trusting incoming blocks, at least temporarily. It
also hurts security to some degree because a portion of the network's
computing power is potentially building on an invalid blockchain,
instead of building on the strongest and valid chain.

The second bullet point has a negative effect on miner decentralization,
see \protect\hyperlink{minerdecentralization}{Miner decentralization},
because usually the pools with the best network connectivity and
bandwidth are also the largest, causing miners to gravitate towards a
few big pools.

\hypertarget{_horizontal_scaling}{%
\subsubsection{Horizontal scaling}\label{_horizontal_scaling}}

Horizontal scaling refers to techniques that divide the workload across
multiple machines. While this is a prevalent scaling approach among
popular websites and databases, it's not easily done in Bitcoin.

Many people refer to this Bitcoin scaling approach as \emph{sharding}.
Basically, it consists in letting each full node verify just a portion
of the blockchain. Peter Todd has put a lot of thought into the concept
of sharding. He wrote a
\href{https://petertodd.org/2015/why-scaling-bitcoin-with-sharding-is-very-hard}{blog
post} explaining sharding in general terms, and also presenting his own
idea called \emph{treechains}. The article is a difficult read, but Todd
makes some points that are quite digestible.

\begin{quote}
In sharded systems the ``full node defense'' doesn't work, at least
directly. The whole point is that not everyone has all the data, so you
have to decide what happens when it's not available.

---  Peter Todd Why Scaling Bitcoin With Sharding Is Very Hard (2015)
\end{quote}

Then he presents various ideas on how to tackle sharding, or horizontal
scaling. Towards the end of the post he concludes:

\begin{quote}
There's a big problem though: holy !@\#\$ is the above complex compared
to Bitcoin! Even the ``kiddy'' version of sharding - my linearization
scheme rather than zk-SNARKS - is probably one or two orders of
magnitude more complex than using the Bitcoin protocol is right now, yet
right now a huge \% of the companies in this space seem to have thrown
their hands up and used centralized API providers instead. Actually
implementing the above and getting it into the hands of end-users won't
be easy.

On the other hand, decentralization isn't cheap: using PayPal is one or
two orders of magnitude simpler than the Bitcoin protocol.

---  Peter Todd Why Scaling Bitcoin With Sharding Is Very Hard (2015)
\end{quote}

The conclusion he makes is that sharding \emph{might} be technically
possible, but it would come at the cost of tremendous complexity. Given
that many users already find Bitcoin too complex and prefer to use
centralized services instead, it's going to be hard to convince them to
use something even more complex.

\hypertarget{_inward_scaling}{%
\subsubsection{Inward scaling}\label{_inward_scaling}}

While horizontal and vertical scaling have historically worked out well
in centralized systems like databases and web servers, they don't seem
to be suitable for a decentralized network like Bitcoin due to their
centralizing effects.

An approach that gets far too little appreciation is what we can call
\emph{inward scaling}, which translates into ``do more with less''. It
refers to the ongoing work constantly done by many developers to
optimize the algorithms already in place, so that we can do more within
the existing limits of the system.

The improvements that have been achieved through inward scaling are
impressive, to say the least. To give you a general idea of the
improvements over the years, Jameson Lopp
\href{https://blog.lopp.net/bitcoin-core-performance-evolution/}{has run
benchmark tests} on blockchain synchronization, comparing many different
versions of Bitcoin Core going back to version 0.8.

\begin{figure}
\centering
\includegraphics{images/Bitcoin-Core-Sync-Performance-1.png}
\caption{Initial block download performance of various versions of
Bitcoin Core. On the Y-axis is the block height synced and on the X-axis
is the time it took to sync to that height. Source:
\url{https://blog.lopp.net/bitcoin-core-performance-evolution/}}
\end{figure}

The different lines represent different versions of Bitcoin Core. The
leftmost line is the latest, i.e. version 0.22, which was released in
September 2021 and took 396 minutes to fully sync. The rightmost one is
version 0.8 from November 2013, which took 3452 minutes. All of this -
roughly 10x - improvement is due to inward scaling.

The improvements could be categorized as either saving space (RAM, disk,
bandwidth, etc.) or saving computational power. Both categories
contribute to the improvements in the diagram above.

A good example of computational improvement can be found in the
\href{https://github.com/bitcoin-core/secp256k1}{libsecp256k1} library,
which, among other things, implements the cryptographic primitives
needed to make and verify digital signatures. Pieter Wuille is one of
the contributors to this library, and he wrote a
\href{https://twitter.com/pwuille/status/1450471673321381896}{Twitter
thread} showcasing the performance improvements achieved through various
pull requests.

\begin{figure}
\centering
\includegraphics{images/libsecp256k1speedups.png}
\caption{Performance of signature verification over time, with
significant pull requests marked on the timeline. Source:
\url{https://twitter.com/pwuille/status/1450471673321381896}}
\end{figure}

The graph shows the trend for two different 64-bit CPU types, namely ARM
and x86. The difference in performance is due to the more specialized
instructions available on x86 compared to the ARM architecture, which
has fewer and more generic instructions. However, the general trend is
the same for both architectures. Note that the Y-axis is logarithmic,
which makes the improvements look less impressive than they actually
are.

There are also several good examples of space-saving improvements that
contributed to performance enhancement. In a
\href{https://murchandamus.medium.com/2-of-3-multisig-inputs-using-pay-to-taproot-d5faf2312ba3}{Medium
blog post} about Taproot's contribution to saving space, user Murch
compares how much block space a 2-of-3 threshold signature would
require, using Taproot in various ways as well as not using it at all.

\begin{figure}
\centering
\includegraphics{images/murch-taproot.png}
\caption{Space savings for different spending types, Taproot and legacy
versions.}
\end{figure}

A 2-of-3 multisig using native Segwit would require a total of 104.5+43
vB = 147.5 vB, whereas the most space-conservative use of Taproot would
require only 57.5+43 vB = 100.5 vB in the standard use case. At worst
and in rare cases, like when a standard signer is not available for some
reason, Taproot would use 107.5+43 vB = 150.5 vB. You don't have to
understand all the details, but this should give you an idea of how
developers think about saving space - every little byte counts.

Apart from inward scaling in Bitcoin software, there are some ways in
which users can contribute to inward scaling, too. They can make their
transactions more intelligently to save on transaction fees while
simultaneously decreasing their footprints on full node requirements.
Two commonly used techniques toward such goal are called transaction
batching and output consolidation.

The idea with transaction batching is to combine multiple payments into
one single transaction, instead of making one transaction per payment.
This can save you a lot of fees, and at the same time reduce the block
space load.

\begin{figure}
\centering
\includegraphics{images/tx-batching.png}
\caption{Transaction batching combines multiple payments into a single
transaction to save on fees.}
\end{figure}

Output consolidation refers to taking advantage of periods of low demand
for block space to combine multiple outputs into a single output. This
can reduce your fee cost later, when you'll need to make a payment while
the demand for block space is high.

\begin{figure}
\centering
\includegraphics{images/utxo-consolidation.png}
\caption{Output consolidation. Melt your coins into one big coin when
fees are low to save fees later.}
\end{figure}

It may not be obvious how output consolidation contributes to inward
scaling. After all, the total amount of blockchain data is even slightly
increased with this method. Nonetheless, the UTXO set, i.e. the database
that keeps track of who owns which coins, shrinks because you spend more
UTXOs than you create. This alleviates the burden for full nodes to
maintain their UTXO sets.

Unfortunately, however, these two techniques of \emph{UTXO management}
could be bad for your own or your payees' privacy. In the batching case,
each payee will know that all the batched outputs are from you to other
payees (except possibly the change). In the UTXO consolidation case, you
will reveal that the outputs you consolidate belong to the same wallet.
So you may have to make a trade-off between cost efficiency and privacy.

\hypertarget{layeredscaling}{%
\subsubsection{Layered scaling}\label{layeredscaling}}

The most impactful approach to scaling is probably layering. The general
idea behind layering is that a protocol can settle payments between
users without adding transactions to the blockchain. This was already
discussed briefly in \protect\hyperlink{trustlessness}{Trustlessness}
and \protect\hyperlink{privacymeasures}{Privacy measures}.

A layered protocol begins with two or more people agreeing on a start
transaction that's put on the blockchain, as illustrated in
\protect\hyperlink{fig-scaling-layer}{figure\_title}.

\begin{figure}
\hypertarget{fig-scaling-layer}{%
\centering
\includegraphics{images/scaling-layer.png}
\caption{A typical layer 2 protocol on top of Bitcoin, layer
1.}\label{fig-scaling-layer}
}
\end{figure}

How this start transaction is created varies between protocols, but a
common theme is that the participants create an unsigned start
transaction and a number of pre-signed punishment transactions, that
spend the output of the start transaction in various ways. Subsequently,
the start transaction is fully signed and published to the blockchain,
and the punishment transactions can be fully signed and published to
punish a misbehaving party. This incentivizes the participants to keep
their promises so that the protocol can work in a trustless way.

Once the start transaction is on the blockchain, the protocol can do
what it's supposed to do. For instance, it could do super fast payments
between participants, implement some privacy-enhancing techniques, or do
more advanced scripting that would not be supported by the Bitcoin
blockchain.

We won't detail how specific protocols work, but as you can see in
\protect\hyperlink{fig-scaling-layer}{figure\_title}, the blockchain is
rarely used during the protocol's life cycle. All the juicy action
happens \emph{off-chain}. We've seen how this can be a win for privacy
if done right, but it can also be an advantage for scalability.

In a
\href{https://www.reddit.com/r/Bitcoin/comments/438hx0/a_trip_to_the_moon_requires_a_rocket_with/}{Reddit
post} titled ``A trip to the moon requires a rocket with multiple stages
or otherwise the rocket equation will eat your lunch\ldots\hspace{0pt}
packing everyone in clown-car style into a trebuchet and hoping for
success is right out.'', Gregory Maxwell explains why layering is our
best shot at getting Bitcoin to scale by orders of magnitudes.

He starts by emphasizing the fallacy in viewing Visa or Mastercard as
Bitcoin's main competitors and highlighting how increasing the maximum
block size is a bad approach to meet said competition. Then he talks
about how to make some real difference by using layers.

\begin{quote}
So-\/- Does that mean that Bitcoin can't be a big winner as a payments
technology? No. But to reach the kind of capacity required to serve the
payments needs of the world we must work more intelligently.

From its very beginning Bitcoin was design to incorporate layers in
secure ways through its smart contracting capability (What, do you think
that was just put there so people could wax-philosophic about
meaningless "DAOs"?). In effect we will use the Bitcoin system as a
highly accessible and perfectly trustworthy robotic judge and conduct
most of our business outside of the court room-\/- but transact in such
a way that if something goes wrong we have all the evidence and
established agreements so we can be confident that the robotic court
will make it right. (Geek sidebar: If this seems impossible, go read
this old post on transaction cut-through)

This is possible precisely because of the core properties of Bitcoin. A
censorable or reversible base system is not very suitable to build
powerful upper layer transaction processing on top of\ldots\hspace{0pt}
and if the underlying asset isn't sound, there is little point in
transacting with it at all.

---  Gregory Maxwell r/Bitcoin on Reddit (2016)
\end{quote}

The analogy with the judge is quite illustrative of how layering works:
this judge must be incorruptible and never change her mind, otherwise
the layers above Bitcoin's base layer will not work reliably.

He continues by making a point about centralized services. There's
usually no problem with trusting a central server with trivial amounts
of Bitcoin to get things done: that's also layered scaling.

Many years have passed since Maxwell wrote the piece above, and his
words still stand correct. The success of the Lightning Network proves
that layering is indeed a way forward to increase the utility of
Bitcoin.

\hypertarget{whenshithitsthefan}{%
\section{When shit hits the fan}\label{whenshithitsthefan}}

\includegraphics{images/shtf-banner.jpg}

Bitcoin is built by people. People write the software, and people then
run this software. When a security vulnerability or a severe bug is
discovered - is there really a distinction between the two? - it's
always discovered by people, flesh and blood. This chapter contemplates
what people do, should, and shouldn't do when shit hits the fan. The
first section explains the term \emph{responsible disclosure}, which
refers to how someone who discovers a vulnerability can act responsibly
to help minimize the damage from it. The rest of the chapter takes you
on a tour through some of the most severe vulnerabilities discovered
over the years, and how they were handled by developers, miners, and
users. Things were not as rigorous in Bitcoin's early childhood as they
are today.

\hypertarget{responsible-disclosure}{%
\subsection{Responsible disclosure}\label{responsible-disclosure}}

Imagine you discover a bug in Bitcoin Core, a bug that allows anyone to
remotely shut down a Bitcoin Core node by using some specially crafted
network messages. Imagine also you are not malicious and would like this
issue to remain unexploited. What do you do? If you remain silent about
it, someone else will probably discover the issue, and you can't be sure
that person won't be malicious.

When a security issue is discovered, the person discovering it should
employ \emph{responsible disclosure} which is a term often used among
Bitcoin developers. The term is
\href{https://en.wikipedia.org/wiki/Coordinated_vulnerability_disclosure}{explained
on Wikipedia}:

\begin{quote}
Developers of hardware and software often require time and resources to
repair their mistakes. Often, it is ethical hackers who find these
vulnerabilities.{[}1{]} Hackers and computer security scientists have
the opinion that it is their social responsibility to make the public
aware of vulnerabilities. Hiding problems could cause a feeling of false
security. To avoid this, the involved parties coordinate and negotiate a
reasonable period of time for repairing the vulnerability. Depending on
the potential impact of the vulnerability, the expected time needed for
an emergency fix or workaround to be developed and applied and other
factors, this period may vary between a few days and several months.

---  Wikipedia Responsible disclosure article
\end{quote}

This means that if you find a security issue, you should report this to
the team responsible for the system. But what does this mean in the
context of Bitcoin? As noted in
\protect\hyperlink{softwaremaintenance}{the Open source chapter}, no one
controls Bitcoin, but there's currently a focal point for Bitcoin
development, namely the
\href{https://github.com/bitcoin/bitcoin}{Bitcoin Core Github
repository}. The maintainers of said repository are responsible for the
code in it, but they're not responsible for the system as a whole - no
one is. Nevertheless, the general best practice is to send an email to
\href{mailto:security@bitcoincore.org}{\nolinkurl{security@bitcoincore.org}}.

In an
\href{https://lists.linuxfoundation.org/pipermail/bitcoin-dev/2017-September/015002.html}{email
thread} titled ``Responsible disclosure of bugs'' from 2017, Anthony
Towns tried to summarize what he perceived to be the current best
practices. He had collected inputs from several sources and different
people to inform his view on the subject.

\begin{quote}
\begin{itemize}
\item
  Vulnerabilities should be reported via security at bitcoincore.org
  {[}0{]}
\item
  A critical issue (that can be exploited immediately or is already
  being exploited causing large harm) will be dealt with by:

  \begin{itemize}
  \item
    a released patch ASAP
  \item
    wide notification of the need to upgrade (or to disable affected
    systems)
  \item
    minimal disclosure of the actual problem, to delay attacks {[}1{]}
    {[}2{]}
  \end{itemize}
\item
  A non-critical vulnerability (because it is difficult or expensive to
  exploit) will be dealt with by:

  \begin{itemize}
  \item
    patch and review undertaken in the ordinary flow of development
  \item
    backport of a fix or workaround from master to the current released
    version {[}2{]}
  \end{itemize}
\item
  Devs will attempt to ensure that publication of the fix does not
  reveal the nature of the vulnerability by providing the proposed fix
  to experienced devs who have not been informed of the vulnerability,
  telling them that it fixes a vulnerability, and asking them to
  identify the vulnerability. {[}2{]}
\item
  Devs may recommend other bitcoin implementations adopt vulnerability
  fixes prior to the fix being released and widely deployed, if they can
  do so without revealing the vulnerability; eg, if the fix has
  significant performance benefits that would justify its inclusion.
  {[}3{]}
\item
  Prior to a vulnerability becoming public, devs will generally
  recommend to friendly altcoin devs that they should catch up with
  fixes. But this is only after the fixes are widely deployed in the
  bitcoin network. {[}4{]}
\item
  Devs will generally not notify altcoin developers who have behaved in
  a hostile manner (eg, using vulnerabilities to attack others, or who
  violate embargoes). {[}5{]}
\item
  Bitcoin devs won't disclose vulnerability details until
  \textgreater80\% of bitcoin nodes have deployed the fixes.
  Vulnerability discovers are encouraged and requested to follow the
  same policy. {[}1{]} {[}6{]}
\end{itemize}

---  Anthony Towns in thread "`Responsible disclosure of bugs`"
Bitcoin-dev email list (2017)
\end{quote}

This list displays how careful one must be when publishing patches for
Bitcoin, since the patch itself might give away the vulnerability. The
fourth bullet is particularly interesting as it explains how to test
whether a patch has been disguised well enough. Indeed, if a few really
experienced developers can't spot the vulnerability even knowing that
the patch fixes one, it will probably be really hard for others to
discover it.

The thread that led to this email was discussing whether, when, and how
to disclose vulnerabilities to altcoins and other implementations of
Bitcoin. There is no clear answer here. ``Helping the good guys'' seems
like the sensible thing to do, but who decides who they are and where
does one draw the line? Bryan Bishop
\href{https://lists.linuxfoundation.org/pipermail/bitcoin-dev/2017-September/014983.html}{argued}
that helping altcoins and even scamcoins defend themselves against
security exploits was a moral duty.

\begin{quote}
It's not enough to defend bitcoin and its users from active threats,
there is a more general responsibility to defend all kinds of users and
different software from many kinds of threats in whatever forms, even if
folks are using stupid and insecure software that you personally don't
maintain or contribute to or advocate for. Handling knowledge of a
vulnerability is a delicate matter and you might be receiving knowledge
with more serious direct or indirect impact than originally described.

---  Bryan Bishop in thread "`Responsible disclosure of bugs`"
Bitcoin-dev email list (2017)
\end{quote}

Also leading up to Town's email above was a
\href{https://lists.linuxfoundation.org/pipermail/bitcoin-dev/2017-September/014977.html}{post}
by Gregory Maxwell, in which he argued that security vulnerabilities
could be more severe than they appear.

\begin{quote}
I've multiple time seen a hard to exploit issue turn out to be trivial
when you find the right trick, or a minor dos issue turn our to far more
serious.

Simple performance bugs, expertly deployed, can potentially be used to
carve up the network-\/-\/- miner A and exchange B go in one partition,
everyone else in another.. and doublespend.

And so on. So while I absolutely do agree that different things should
and can be handled differently, it is not always so clear cut. It's
prudent to treat things as more severe than you know them to be.

---  Gregory Maxwell in thread "`Responsible disclosure of bugs`"
Bitcoin-dev email list (2017)
\end{quote}

So, even if a vulnerability seems hard to exploit, it might be best to
assume that it's easily exploitable and you just haven't figured out how
yet.

He also mentions how ``it's somewhat incorrect to call this thread
anything about disclosure, this thread is not about disclosure.
Disclosure is when you tell the vendor. This thread is about publication
and that has very different implications. Publication is when you're
sure you've told the prospective attackers''. This last observation
concerning the distinction between disclosure and publication is an
important one. The easy part is responsible disclosure; the hard part is
sensible publishing.

\hypertarget{_traumatic_childhood}{%
\subsection{Traumatic childhood}\label{_traumatic_childhood}}

Bitcoin started out as a one-man (at least that's what its creator's
pseudonym suggests) project, and bitcoin had initially little to no
value. As such, vulnerabilities and bug fixes were not as rigorously
handled as they are today.

The Bitcoin wiki has a
\href{https://en.bitcoin.it/wiki/Common_Vulnerabilities_and_Exposures}{list
of common vulnerabilities and exposures} (CVEs) that Bitcoin has gone
through. This section constitutes a little exposé of some of the
security issues and incidents from the early years of Bitcoin. We won't
cover them all, but we selected a few that we find especially
interesting.

\hypertarget{cve-2010-5141}{%
\subsubsection{2010-07-28: Spend anyone's coins
(CVE-2010-5141)}\label{cve-2010-5141}}

On July 28, 2010, a pseudonymous person by the name ArtForz discovered a
bug in version 0.3.4 that would let anyone take coins from anyone else.
ArtForz \emph{responsibly} reported this to Satoshi Nakamoto and to
another Bitcoin developer named Gavin Andresen.

The problem was that the script operator \texttt{OP\_RETURN} would
simply exit the program execution, so if the scriptPubKey was
\texttt{\textless{}pubkey\textgreater{}\ OP\_CHECKSIG} and scriptSig was
\texttt{OP\_1\ OP\_RETURN}, the part of the program in the scriptPubKey
would never execute. The only thing that would happen would be for
\texttt{1} to be put on the stack and then \texttt{OP\_RETURN} would
cause the program to exit. Any non-zero value on top of the stack after
the program has executed means that the spending condition is fulfilled.
Since the top stack element \texttt{1} is non-zero, the spending would
be OK.

This was the code for handling of \texttt{OP\_RETURN}:

\begin{verbatim}
            case OP_RETURN:
            {
                pc = pend;
            }
            break;
\end{verbatim}

The effect of \texttt{pc\ =\ pend;} was for the rest of the program to
get skipped, meaning that any locking script in scriptPubKey would be
ignored. The fix consisted in changing the meaning of
\texttt{OP\_RETURN} so that it immediately failed, instead.

\begin{verbatim}
            case OP_RETURN:
            {
                return false;
            }
            break;
\end{verbatim}

Satoshi made this change locally and built an executable binary with
version 0.3.5 from it. Then he posted on Bitcointalk forum ``*** ALERT
*** Upgrade to 0.3.5 ASAP'', urging users to install this binary version
of his, without presenting the source code for it.

\begin{quote}
Please upgrade to 0.3.5 ASAP! We fixed an implementation bug where it
was possible that bogus transactions could be accepted. Do not accept
Bitcoin transactions as payment until you upgrade to version 0.3.5!

---  Satoshi Nakamoto Bitcointalk forum (2010)
\end{quote}

The original message was later edited and is no longer available in its
full form. The above snippet is from a
\href{https://bitcointalk.org/index.php?topic=626.msg6458\#msg6458}{quoting
answer}. Some users tried Satoshi's binary, but ran into issues with it.
Shortly after,
\href{https://bitcointalk.org/index.php?topic=626.msg6469\#msg6469}{Satoshi
wrote}:

\begin{quote}
Haven't had time to update the SVN yet. Wait for 0.3.6, I'm building it
now. You can shut down your node in the meantime.

---  Satoshi Nakamoto Bitcointalk forum (2010)
\end{quote}

And 35 minutes later,
\href{https://bitcointalk.org/index.php?topic=626.msg6480\#msg6480}{he
wrote}

\begin{quote}
SVN is updated with version 0.3.6.

Uploading Windows build of 0.3.6 to Sourceforge now, then will rebuild
linux.

---  Satoshi Nakamoto Bitcointalk forum (2010)
\end{quote}

At this point he also seemed to have updated the original post to
mention 0.3.6 instead of 0.3.5:

\begin{quote}
Please upgrade to 0.3.6 ASAP! We fixed an implementation bug where it
was possible that bogus transactions could be displayed as accepted. Do
not accept Bitcoin transactions as payment until you upgrade to version
0.3.6!

If you can't upgrade to 0.3.6 right away, it's best to shut down your
Bitcoin node until you do.

Also in 0.3.6, faster hashing:\\
- midstate cache optimisation thanks to tcatm\\
- Crypto++ ASM SHA-256 thanks to BlackEye\\
Total generating speedup 2.4x faster.

Download:\\
\url{http://sourceforge.net/projects/bitcoin/files/Bitcoin/bitcoin-0.3.6/}

Windows and Linux users: if you got 0.3.5 you still need to upgrade to
0.3.6.

---  Satoshi Nakamoto Bitcointalk forum (2010)
\end{quote}

Note the difference in the characterization of the problem from the
first message: ``could be displayed as accepted'' vs ``could be
accepted''. Maybe Satoshi downplayed the severity of the bug in his
communication so as not to draw too much attention to the actual issue.
Anyhow, people upgraded to 0.3.6 and it worked as expected. This
particular issue was resolved, amazingly, with no bitcoin losses.

Satoshi's message also described some performance optimization for
mining. It's unclear why that was included in a critical security fix,
it's possible that the purpose was to obfuscate the real issue. However,
it seems more likely that he just released whatever was on the head of
the development branch of the Subversion repository, with the security
fix added to it.

At that time, there weren't nearly as many users as there are today, and
bitcoin's value was close to zero. If this bug response was played out
today, it would be considered a complete shit-show for multiple reasons:

\begin{itemize}
\item
  Satoshi made a binary-only release of 0.3.5 containing the fix. No
  patch or code was provided, maybe as a measure to obfuscate the issue.
\item
  0.3.5
  \href{https://bitcointalk.org/index.php?topic=626.msg6455\#msg6455}{didn't
  even work}.
\item
  The fix in 0.3.6 was actually a hard fork, as explained in
  \protect\hyperlink{historic-upgrades}{Historic upgrades}.
\end{itemize}

Another debatable thing is whether it's good or bad that users were
asked to shut down their nodes. This wouldn't be doable today, but at
that time lots of users were actively following the forums for updates
and were usually on top of things. Given that it was possible to do
this, it might have been a sensible thing to do.

\hypertarget{combined-output-overflow}{%
\subsubsection{2010-08-15 Combined output overflow
(CVE-2010-5139)}\label{combined-output-overflow}}

In mid-August 2010, Bitcointalk forum user jgarzik, a.k.a. Jeff Garzik,
\href{https://bitcointalk.org/index.php?topic=822.msg9474\#msg9474}{discovered
that} a certain transaction at block height 74638 had two outputs of
unusually high value:

\begin{quote}
The "value out" in this block \#74638 is quite strange:

\begin{verbatim}
...
  "out" : [
      {
          "value" : 92233720368.54277039,
          "scriptPubKey" : "OP_DUP OP_HASH160 0xB7A73EB128D7EA3D388DB12418302A1CBAD5E890 OP_EQUALVERIFY OP_CHECKSIG"
      },
      {
          "value" : 92233720368.54277039,
          "scriptPubKey" : "OP_DUP OP_HASH160 0x151275508C66F89DEC2C5F43B6F9CBE0B5C4722C OP_EQUALVERIFY OP_CHECKSIG"
      }
  ]
...
\end{verbatim}

92233720368.54277039 BTC? Is that UINT64\_MAX, I wonder?

---  Jeff Garzik Bitcointalk forum (2010)
\end{quote}

Presumably, there was a bug causing two int64 (not uint64, as Garzik
supposed) outputs' sum to overflow to a negative value -0.00997538 BTC.
Whatever the sum of the inputs, the ``sum'' of the outputs would be
smaller, making this transaction OK according to the code at the time.

In this case, the bug had been disclosed and published through an actual
exploit. An unfortunate outcome of this was that about 2x92 billion
bitcoin had been created, which severely diluted the money supply of
around 3.7 million coins that existed at that time.

In a related thread,
\href{https://bitcointalk.org/index.php?topic=823.msg9531\#msg9531}{Satoshi
posted} that he'd appreciate it if people stopped mining (or
\emph{generating}, as they called it back then).

\begin{quote}
It would help if people stop generating. We will probably need to re-do
a branch around the current one, and the less you generate the faster
that will be.

A first patch will be in SVN rev 132. It's not uploaded yet. I'm pushing
some other misc changes out of the way first, then I'll upload the patch
for this.

---  Satoshi Nakamoto Bitcointalk forum (2010)
\end{quote}

His plan was to make a soft fork to make transactions like the one
discussed here invalid, thus invalidating the blocks (especially block
74638) that contained such transactions. Less than an hour later, he
committed a \href{https://sourceforge.net/p/bitcoin/code/132/}{patch in
revision 132} of the Subversion repository and
\href{https://bitcointalk.org/index.php?topic=823.msg9548\#msg9548}{posted
to the forum} describing what he thought users should do:

\begin{quote}
Patch is uploaded to SVN rev 132!

For now, recommended steps:\\
1) Shut down.\\
2) Download knightmb's blk files. (replace your blk0001.dat and
blkindex.dat files)\\
3) Upgrade.\\
4) It should start out with less than 74000 blocks. Let it redownload
the rest.

If you don't want to use knightmb's files, you could just delete your
blk*.dat files, but it's going to be a lot of load on the network if
everyone is downloading the whole block index at once.

I'll build releases shortly.
\end{quote}

He wanted people to download block data from a specific user, namely
knightmb, who had published his blockchain as it appeared on his disk,
the files blkXXXX.dat and blkindex.dat. The reason for downloading the
blockchain data this way, as opposed to synchronizing from scratch, was
to reduce network bandwidth bottlenecks.

There was a big caveat with this: the data users would download from
knightmb \href{https://bitcoin.stackexchange.com/a/113682/69518}{weren't
verified by the Bitcoin software} at startup. The blkindex.dat file
contained the UTXO set, and the software would accept any data therein
as if it had already verified it. knightmb could have manipulated the
data to give himself or anyone else some bitcoins.

Again, people seemed to go along with this, and the reversal of the
invalid block and its successors was successful. Miners started working
on a new successor to block
\href{https://mempool.space/block/0000000000606865e679308edf079991764d88e8122ca9250aef5386962b6e84}{74637}
and, according to the block's timestamp, a successor appeared at 23:53
UTC, about 6 hours after the issue was discovered. At 08:10 the
following day, on August 16, around block 74689, the new chain had
overtaken the old chain, therefore all non-upgraded nodes reorged to
follow the new chain. This is the deepest reorg - 52 blocks - in
Bitcoin's history.

Compared to the OP\_RETURN issue, this issue was handled in a somewhat
cleaner way:

\begin{itemize}
\item
  No binary-only patch release
\item
  The released software worked as intended
\item
  No hard fork
\end{itemize}

Users were asked to stop mining during this issue as well. We can
discuss whether this is a good idea or not, but imagine you're a miner
and you're convinced that any blocks on top of the bad block will
eventually get wiped out in a deep reorg: why would you waste resources
on mining doomed blocks?

You might also think that it's a bit fishy to do as suggested by
Nakamoto and download the blockchain, including the UTXO set, from a
random dude's hard drive. If so, you're right: that is fishy. But, given
the circumstances, this emergency response was a sensible one.

There's an important difference between this case and the previous
OP\_RETURN case: this issue was exploited in the wild, and thus a fix
could be made more straightforward. In the case of OP\_RETURN, they had
to obfuscate the fix and make public statements that didn't directly
reveal what the issue was.

\hypertarget{march2013split}{%
\subsubsection{2013-03-11 DB locks issue 0.7.2 - 0.8.0
(CVE-2013-3220)}\label{march2013split}}

A very interesting an educationally valuable issue surfaced in March
2013. It appeared that the blockchain had split (although the word
``fork'' is used in the quote below) after block 225429. The details of
this incident were
\href{https://github.com/bitcoin/bips/blob/master/bip-0050.mediawiki}{reported
in BIP50}. The summary says:

\begin{quote}
A block that had a larger number of total transaction inputs than
previously seen was mined and broadcasted. Bitcoin 0.8 nodes were able
to handle this, but some pre-0.8 Bitcoin nodes rejected it, causing an
unexpected fork of the blockchain. The pre-0.8-incompatible chain (from
here on, the 0.8 chain) at that point had around 60\% of the mining hash
power ensuring the split did not automatically resolve (as would have
occurred if the pre-0.8 chain outpaced the 0.8 chain in total work,
forcing 0.8 nodes to reorganise to the pre-0.8 chain).

In order to restore a canonical chain as soon as possible, BTCGuild and
Slush downgraded their Bitcoin 0.8 nodes to 0.7 so their pools would
also reject the larger block. This placed majority hashpower on the
chain without the larger block, thus eventually causing the 0.8 nodes to
reorganise to the pre-0.8 chain.

---  Various Bitcoin Core developers BIP50 (2013)
\end{quote}

The quick action that the mining pools BTCGuild and Slush took was
imperative in this emergency. They were able to tip the majority of the
hash power over to the pre-0.8 branch of the split, and thus help
restore consensus. This gave developers the time to figure out a
sustainable fix.

What's also very interesting in this issue is that version 0.7.2 was
incompatible with itself, as was the case with prior versions too. This
is explained in the
\href{https://github.com/bitcoin/bips/blob/master/bip-0050.mediawiki\#root-cause}{Root
cause section of BIP50}:

\begin{quote}
With the insufficiently high BDB lock configuration, it implicitly had
become a network consensus rule determining block validity (albeit an
inconsistent and unsafe rule, since the lock usage could vary from node
to node).

---  Various Bitcoin Core developers BIP50 (2013)
\end{quote}

In short, the issue is that the number of database locks the Bitcoin
Core software needs to verify a block is not deterministic. One node
might need X locks while another node might need X+1 locks. The nodes
also have a limit on how many locks Bitcoin can take. If the number of
locks needed exceeds the limit, the block will be considered invalid. So
if X+1 exceeds the limit but not X, then the two nodes will split the
blockchain and disagree on which branch is valid.

The solution chosen, apart from the immediate actions taken by the two
pools to restore consensus, was to

\begin{itemize}
\item
  limit the blocks in terms of both size and locks needed on version
  0.8.1
\item
  patch old versions (0.7.2 and some older ones) with the same new
  rules, and increase the global lock limit.
\end{itemize}

Except for the increased global lock limit in the second bullet, these
rules were implemented temporarily for a pre-determined amount of time.
The plan was to remove these limits once most nodes had upgraded.

This soft fork dramatically reduced the risk of consensus failure, and a
few months later, on May 15, the temporary rules were deactivated in
concert across the network. Note that this deactivation was in effect a
hard fork, but it was not contentious. Furthermore, it was released
along with the preceding soft fork, so people running the soft-forked
software were well aware that a hard fork would follow it. Therefore,
the vast majority of nodes remained in consensus when the hard fork got
activated. Unfortunately, though, a few nodes that didn't upgrade were
lost in the process.

One might wonder if this would be doable today. The mining landscape is
more complex today, and, depending on the hash power on each side of the
split, it might be hard to roll out a patch such as the one in BIP50
quickly enough. It'd probably be hard to convince miners on the
``wrong'' branch to let go of their block rewards.

\hypertarget{bip66-splits}{%
\subsubsection{BIP66}\label{bip66-splits}}

BIP66 is interesting because it highlights the importance of

\begin{itemize}
\item
  good selection cryptography
\item
  responsible disclosure
\item
  deployment without revealing the vulnerability
\item
  mining on top of verified blocks
\end{itemize}

BIP66 was a proposal to tighten up the rules for signature encodings in
Bitcoin Script. The
\href{https://github.com/bitcoin/bips/blob/master/bip-0066.mediawiki\#motivation}{motivation}
was to be able to parse signatures with software or libraries other than
OpenSSL and even recent versions of OpenSSL. OpenSSL is a library for
general purpose cryptography that Bitcoin Core used at that time.

The BIP activated on July 4, 2015. However, while the above is true,
BIP66 also fixes a much more severe issue not mentioned in the BIP.

\hypertarget{_the_vulnerability}{%
\paragraph{The vulnerability}\label{_the_vulnerability}}

The full disclosure of this issue was published on July 28 2015 by
Pieter Wuille in an
\href{https://lists.linuxfoundation.org/pipermail/bitcoin-dev/2015-July/009697.html}{email
to the Bitcoin-dev mailing list}:

\begin{quote}
Hello all,

I'd like to disclose a vulnerability I discovered in September 2014,
which became unexploitable when BIP66's 95\% threshold was reached
earlier this month.

\#\# Short description:

A specially-crafted transaction could have forked the blockchain between
nodes:

\begin{itemize}
\item
  using OpenSSL on a 32-bit systems and on 64-bit Windows systems
\item
  using OpenSSL on non-Windows 64-bit systems (Linux, OSX,
  \ldots\hspace{0pt})
\item
  using some non-OpenSSL codebases for parsing signatures
\end{itemize}

---  Pieter Wuille on Bitcoin-dev mailing list Disclosure: consensus bug
indirectly solved by BIP66 (2015)
\end{quote}

The email further lays out the details about how the issue got
discovered and more exactly what caused it. At the end, he submits a
timeline of the events, and we will replay some of the most important
ones here. Some of them have, as illustrated by
\protect\hyperlink{fig-bip66-timeline-1}{figure\_title}, already been
described.

\begin{figure}
\hypertarget{fig-bip66-timeline-1}{%
\centering
\includegraphics{images/bip66-timeline-1.png}
\caption{Timeline of events surrounding BIP66. Items in black have been
explained above.}\label{fig-bip66-timeline-1}
}
\end{figure}

\hypertarget{_before_discovery}{%
\paragraph{Before discovery}\label{_before_discovery}}

Without anyone knowing about the issue, it could have been resolved by
the now widthdrawn BIP62, which was a proposal to reduce the
possibilities of transaction malleability. Among the proposed changes in
BIP62 were tightening of the consensus rules for the encoding of
signatures, or ``strict DER encoding''. Pieter Wuille proposed some
tweaks to the BIP in July 2014, that would have solved the issue:

\begin{quote}
\begin{itemize}
\item
  2014-Jul-18: In order to make Bitcoin's signature encoding rules not
  depend on OpenSSL's specific parser, I modified the BIP62 proposal to
  have its strict DER signatures requirement also apply to version 1
  transactions. No non-DER signatures were being mined into blocks
  anymore at the time, so this was assumed to not have any impact. See
  \url{https://github.com/bitcoin/bips/pull/90} and
  \url{http://lists.linuxfoundation.org/pipermail/bitcoin-dev/2014-July/006299.html}.
  Unknown at the time, but if deployed this would have solved the
  vulnerability.
\end{itemize}

---  Pieter Wuille on Bitcoin-dev mailing list Disclosure: consensus bug
indirectly solved by BIP66 (2015)
\end{quote}

Due to the breadth of this BIP, which covered substantially more than
just ``strict DER encoding'', it was constantly changing and never got
near deployment. The BIP was later withdrawn because Segregated Witness,
BIP141, solved transaction malleability in a different and more complete
way.

\hypertarget{_after_discovery}{%
\paragraph{After discovery}\label{_after_discovery}}

OpenSSL released new versions of their software with patches that, if
used in Bitcoin since the beginning, would have solved the issue.
However, using any new version of OpenSSL only in a new release of
Bitcoin Core would make matters worse. Gregory Maxwell explains this in
another
\href{https://lists.linuxfoundation.org/pipermail/bitcoin-dev/2015-January/007097.html}{email
thread} in January 2015:

\begin{quote}
While for most applications it is generally acceptable to eagerly reject
some signatures, Bitcoin is a consensus system where all participants
must generally agree on the exact validity or invalidity of the input
data. In a sense, consistency is more important than ``correctness''.

\ldots\hspace{0pt}

The patches above, however, only fix one symptom of the general problem:
relying on software not designed or distributed for consensus use (in
particular OpenSSL) for consensus-normative behavior. Therefore, as an
incremental improvement, I propose a targeted soft-fork to enforce
strict DER compliance soon, utilizing a subset of BIP62.

---  Gregory Maxwell on OpenSSL upgrade Bitcoin-dev mailing list
\end{quote}

He points out that using code that's not intended for use in consensus
systems poses serious risks, and proposes that Bitcoin implements strict
DER encoding. This is a very clear example of the importance of good
selection cryptography, a term we discussed in
\protect\hyperlink{selectioncryptography}{Selection cryptography}.

These events might give you the impression that Gregory Maxwell knew
about the vulnerability Pieter Wuille later published, but wanted to
help sneak in a fix disguised as a precaution measure, without drawing
too much attention to the actual problem. It might be so, but it's
purely speculation.

Then, as proposed by Maxwell, BIP66 was created as a subset of BIP62
that specified only strict DER encoding. This BIP was apparently broadly
accepted and deployed in July, albeit two blockchain splits ironically
occurred due to \emph{validationless mining}. These splits are discussed
in the next section.

\includegraphics{images/bip66-timeline-2.png}

A key takeaway from this is that BIPs should be more or less
\emph{atomic}, meaning that they should be complete enough to provide
something useful or solve a specific problem, but small enough to allow
for broad support among users. The more stuff you put into a BIP, the
smaller the chance of acceptance.

\hypertarget{bip66splits}{%
\paragraph{Splits due to validationless mining}\label{bip66splits}}

Unfortunately, the story of BIP66 didn't end there. When BIP66 was
activated, it turned out quite messy because some miners didn't verify
the blocks they were trying to extend. This is called validationless
mining, or SPV-mining (as in Simplified Payment Verification). An alert
message was sent out to Bitcoin nodes with a link to
\href{https://bitcoin.org/en/alert/2015-07-04-spv-mining}{a web page
describing the issue}.

\begin{quote}
Early morning on 4 July 2015, the 950/1000 (95\%) threshold was reached.
Shortly thereafter, a small miner (part of the non-upgraded 5\%) mined
an invalid block--as was an expected occurrence. Unfortunately, it
turned out that roughly half the network hash rate was mining without
fully validating blocks (called SPV mining), and built new blocks on top
of that invalid block.

---  Bitcoin Core developers Alert information on bitcoin.org (2015)
\end{quote}

The alert page instructed people to wait for 30 additional confirmations
than they normally would in case they were using older versions of
Bitcoin Core.

The split mentioned above occurred on 2015-07-04 at 02:10 UTC after
block height
\href{https://mempool.space/block/000000000000000006a320d752b46b532ec0f3f815c5dae467aff5715a6e579e}{363730}.
This issue got resolved at 03:50 the same day, after 6 invalid blocks
had been mined. Unfortunately, the same issue happened again the next
day, i.e. on 2015-07-05 at 21:50, but this time the invalid branch only
lasted 3 blocks.

\includegraphics{images/bip66-timeline-3.png}

The events that led up to BIP66, its deployment, and the aftermath are a
very good case study for how careful Bitcoin developers have to be. A
few key takeaways from BIP66:

\begin{itemize}
\item
  The balance between openness and not publishing a vulnerability is a
  delicate one.
\item
  Deploying fixes for non-published vulnerabilities is a tricky game to
  play.
\item
  Retaining consensus is hard.
\item
  Software not intended for consensus systems are generally risky.
\item
  BIPs should be somewhat atomic.
\end{itemize}

\hypertarget{appendixdiscussion}{%
\section{Discussion Questions}\label{appendixdiscussion}}

These discussion questions are not just a recap of the content in
``Bitcoin development philosophy'', they are meant to encourage you to
research further so make sure to go out and explore.

\hypertarget{_decentralization}{%
\subsection{Decentralization}\label{_decentralization}}

\begin{itemize}
\item
  Decentralization is hard. Why do we go through all of this hassle to
  make it work? Could we opt for a hybrid approach, where some parts are
  centralized and others aren't?
\item
  Does decentralization introduce the double spending problem, or does
  the double spending problem require decentralization? How did Satoshi
  solve the double spending problem?
\item
  In which aspects is Bitcoin still most prone to censorship, and why is
  censorship such a bad thing? Are there any arguments in favor of
  censorship?
\item
  It is stated that Bitcoin is permissionless. Are there any other
  payment methods you could consider permissionless?
\end{itemize}

\hypertarget{_trustlessness}{%
\subsection{Trustlessness}\label{_trustlessness}}

\begin{itemize}
\item
  Trustlessness is often a spectrum, not binary. Which aspects of
  Bitcoin are rather trustless, and which typically involve a higher
  level of trust? Can they be mitigated?
\item
  You want to run a full node to be able to fully validate all
  transactions. You download Bitcoin Core from
  \url{https://bitcoin.org/en/download}. Where did you place trust, and
  where are you fully trustless?
\item
  Can you build a trustless system on top of a trusted system?
\end{itemize}

\hypertarget{_privacy}{%
\subsection{Privacy}\label{_privacy}}

\begin{itemize}
\item
  What are some important benefits a user gains when he maintains good
  privacy when interacting with Bitcoin? What are some altruistic
  benefits for the network?
\item
  How does reusing addresses affect your privacy?
\item
  Bitcoin uses a UTXO model, whereas some alternative cryptocurrencies
  use an account model. What are the implications of this choice on
  privacy?
\end{itemize}

\hypertarget{_finite_supply}{%
\subsection{Finite supply}\label{_finite_supply}}

\begin{itemize}
\item
  What is the relation between Bitcoin's finite supply and its coin
  issuance through the coinbase transaction? What is the relation
  between coin issuance and security budget, and how are they at odds?
\item
  What parameters could Satoshi have tweaked to change Bitcoin's supply
  cap? What would change if he had decided to cap the supply to 1
  million? What about 1 trillion?
\item
  Why are some people advocating for an increase in Bitcoin supply? Do
  you think this will happen?
\end{itemize}

\hypertarget{_upgrading}{%
\subsection{Upgrading}\label{_upgrading}}

\begin{itemize}
\item
  What is Speedy Trial and why was it necessary to activate Taproot?
\item
  Why do we need such a high percentage of miners to upgrade in a
  softfork? Why is the threshold not just 51\%?
\end{itemize}

\hypertarget{_adversarial_thinking}{%
\subsection{Adversarial thinking}\label{_adversarial_thinking}}

\begin{itemize}
\item
  What is a sybil attack, and what makes a decentralized network so
  prone to it?
\item
  Why is it important that all players in the Bitcoin network - and not
  just developers - think adversarially?
\end{itemize}

\hypertarget{_open_source}{%
\subsection{Open source}\label{_open_source}}

\begin{itemize}
\item
  Only a handful of maintainers have the necessary GitHub permissions to
  merge code into into the
  \href{https://github.com/bitcoin/bitcoin}{Bitcoin Core} repository.
  Isn't that at odds with a permissionless network?
\item
  Is the open source development process prone to a sybil attack? If so,
  how would you counter that?
\item
  What are the benefits and downsides of relying on third party open
  source libraries, and what is the approach taken with Bitcoin Core?
\item
  In which ways do we need review beyond just code review? How to
  determine how much review is enough?
\item
  How do we ensure there will always be sufficient people with expertise
  working on Bitcoin? What happens when there aren't, and how do we
  asses their integrity and intentions?
\end{itemize}

\hypertarget{_scaling}{%
\subsection{Scaling}\label{_scaling}}

\begin{itemize}
\item
  It is argued that sharding offers scaling benefits at the cost of
  complexity. Why should we or should we not adopt technological
  improvements because they are difficult to understand, even if they
  appear technologically sound?
\item
  What are some examples of inward scaling methods introduced in
  Bitcoin?
\item
  Why is vertical scaling much more difficult in a decentralized system?
  What about horizontal scaling?
\item
  We don't seem to be anywhere near having consensus on how we could
  onboard the entire world onto Bitcoin. Shouldn't Satoshi have at least
  thought of a path of getting there, before mining the first block in
  2009?
\item
  How would you classify (vertical, horizontal, inward, or not a scaling
  technique) each of the following: sharding, blocksize increase,
  SegWit, SPV nodes, centralized exchanges, Lightning Network, block
  interval decrease, Taproot, sidechains
\end{itemize}
